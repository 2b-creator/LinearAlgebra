\chapter{初等矩阵}
\begin{center}
	% \textcolor[RGB]{255, 0, 0}{\faHeart}所以生命啊,它苦涩如歌.\textcolor[RGB]{255, 0, 0}{\faHeart}
	「只愿君心似我心,定不负相思意」
\end{center}
\rightline{——《卜算子$\cdot$我住长江头》}
\vspace{-5pt}
\begin{center}
	\pgfornament[width=0.36\linewidth,color=lsp]{88}
\end{center}

\section{线性方程组的解}

\subsection{二元线性方程组}

考虑一般的二元线性方程组\begin{numcases}{}
	a_{11}x_1+a_{12}x_2=b_1 \label{eq:1-1}\\
	a_{21}x_1+a_{22}x_2=b_2 \label{eq:1-2}
\end{numcases}按照正常的方法,我们将式\ref{eq:1-1}乘系数$a_{21}$,将\ref{eq:1-2}乘系数$a_{11}$后,用变换过的\ref{eq:1-1}减去\ref{eq:1-2}后我们可以得到$$\left( a_{11}a_{22}-a_{12}a_{21} \right)x_1=b_1a_{22}-b_2a_{12}$$我们可以得到方程的解$$x_1=\frac{\begin{vmatrix}
	b_1 & a_{12}\\
	b_2 & a_{22}
\end{vmatrix}}{\begin{vmatrix}
	a_{11} & a_{12}\\
	a_{21} & a_{22}
\end{vmatrix}}$$通过上述方法,我们可以得到$$x_2=\frac{\begin{vmatrix}
	a_{11} & b_1\\
	a_{21} & b_2
\end{vmatrix}}{\begin{vmatrix}
	a_{11} & a_{12}\\
	a_{21} & a_{22}
\end{vmatrix}}$$

\subsection{Cramer 法则}

以上面的二元线性方程组为例我们抽象出的两个解的表示方法,其中我们有系数矩阵行列式$$\mathbf{D}=\begin{vmatrix}
	a_{11} & a_{12}\\
	a_{21} & a_{22}
\end{vmatrix}$$其中常数矩阵$\mathbf{B}=\begin{pmatrix}
	b_1\\
	b_2
\end{pmatrix}$,我们将常数矩阵替换系数矩阵的第$i$列,得到$\mathbf{D}_i$行列式,例如$$\mathbf{D}_1=\begin{vmatrix}
	b_1 &a_{12}\\
	b_2 &a_{22}
   \end{vmatrix},\mathbf{D}_2=\begin{vmatrix}
	a_{11} & b_1 \\
	a_{12} & b_2
\end{vmatrix}$$最后方程组的第$i$个解为$$x_i=\frac{\mathbf{D}_i}{\mathbf{D}}$$

\begin{example}
	使用Cramer法则计算线性方程组$$\left\{\begin{matrix} 
		2x+3y+4z =3 \\  
		5x+4z=2\\
		x+3y+z=1
	  \end{matrix}\right. $$
	  \tcblower
	  \textcolor{purple}{\textbf{解}}:考虑系数矩阵行列式$$\mathbf{D}=\begin{vmatrix}
		2 & 3 & 4\\
		5 & 0 & 4\\
		1 & 3 & 1
	   \end{vmatrix}=33$$常数矩阵$$\mathbf{B}=\begin{pmatrix}
		3 \\
		2 \\
		1
		\end{pmatrix}$$使用常数矩阵替换行列式的第$i$列,我们可以得到$$\mathbf{D}_1=\begin{vmatrix}
			3 & 3 & 4\\
			2 & 0 & 4\\
			1 & 3 & 1
		   \end{vmatrix}=-6,\mathbf{D}_2=\begin{vmatrix}
			2 & 3 & 4\\
			5 & 2 & 4\\
			1 & 1 & 1
		   \end{vmatrix}=5,\mathbf{D}_3=\begin{vmatrix}
			2 & 3 & 3\\
			5 & 0 & 2\\
			1 & 3 & 1
		   \end{vmatrix}=24$$那么该方程组的三个解分别为$$x_1=\frac{\mathbf{D}_1}{\mathbf{D}}=-\frac{2}{11},x_2=\frac{\mathbf{D}_2}{\mathbf{D}}=\frac{5}{33},x_3=\frac{\mathbf{D}_3}{\mathbf{D}}=-\frac{8}{11}$$
\end{example}

\begin{ascolorbox1}{思考}
	当系数矩阵的行列式$\mathbf{D}=0$时是什么情况?
\end{ascolorbox1}

Cramer 法则的适用前提是系数矩阵的行列式值不能为0,读者可以发现若系数矩阵的行列式为0的时候会有一组方程组线性相关,其所面临的情况为有无穷多解或无解,例如$$\left\{\begin{matrix} 
	2x_1+3x_2 = 4 \\  
	4x_1+6x_2 = 8
  \end{matrix}\right. $$这种情况就是无穷多解,而$$\left\{\begin{matrix} 
	2x_1+3x_2 = 4 \\  
	4x_1+6x_2 = 9
  \end{matrix}\right. $$则是无解。

\subsection{齐次线性方程组}

齐次线性方程组是线性方程组中的一个特例,定义如下
\begin{definition}{齐次线性方程组(homogeneous linear equations)}
	齐次线性方程组指的是常数项全部为零的线性方程组,它们一般记作$$\left\{\begin{matrix} 
		a_{11}x_1+a_{12}x_2+a_{13}x_3+\cdots+a_{1n}x_n=0 \\  
		a_{21}x_1+a_{22}x_2+a_{23}x_3+\cdots+a_{2n}x_n=0 \\  
		a_{31}x_1+a_{32}x_2+a_{33}x_3+\cdots+a_{3n}x_n=0 \\
		\cdots \\
		a_{m1}x_1+a_{m2}x_2+a_{m3}x_3+\cdots+a_{mn}x_n=0
	  \end{matrix}\right. $$
\end{definition}

我们在第2章的B组练习中有提到这种方程组,先说结论,当$n>m$的时候齐次线性方程组必有非零解,在练习中,我们使用线性映射$$T(x_1,x_2,\cdots,x_n)=\left( \sum_{i=1}^{n}a_{1i}x_i,\sum_{i=1}^{n}a_{2i}x_i,\cdots,\sum_{i=1}^{n}a_{mi}x_i \right)$$这里的线性映射$T$将$\mathbb{F}^n$映射为$\mathbb{F}^m$,根据线性映射的基本定理有$$\text{dim}\mathbb{F}^n=\text{dim}~\text{null}T+\text{dim}~\text{range}T$$使得齐次线性方程组有无穷多解的充要条件是$T$不是单射;所以$\text{null}T\neq \left\{ 0 \right\}$(否则对应映射到零空间的向量只有$\boldsymbol{0}$)。

接下来我们考虑特殊的情况,当$m=n$的时候,系数矩阵的行列式$\mathbf{D}=\begin{vmatrix}  
	a_{11}& a_{12}& \cdots  & a_{1n} \\  
	a_{21}& a_{22}& \cdots  & a_{2n} \\  
	\vdots & \vdots & \ddots & \vdots \\  
	a_{m1}& a_{m2}& \cdots  & a_{mn}  
  \end{vmatrix}  
$如果我们对其使用 Cramer 法则,当$\mathbf{D}\neq 0$时,$\displaystyle x_i=\frac{\mathbf{D}_i}{\mathbf{D}}=0$有且仅有全为0解,即$x_1=x_2=\cdots=x_n=0$。那如果$\mathbf{D}= 0$则系数矩阵的几行向量线性相关,根据线性相关的向量可以通过去除$i$个向量可以得到线性无关的向量,由于$m=n,m-i<n$,所以我们得到齐次线性方程组有无穷多解。

\section{初等矩阵}

\subsection{增广矩阵}

%cite
首先需要注意的是,按照初等矩阵的定义,增广矩阵不算是初等矩阵的一种,但是因为它可以使用矩阵的初等行变化,所以笔者将其纳入初等矩阵这一节。

考虑线性方程组$$\left\{\begin{matrix} 
	a_{11}x_1+a_{12}x_2+a_{13}x_3+\cdots+a_{1n}x_n=b_1 \\  
	a_{21}x_1+a_{22}x_2+a_{23}x_3+\cdots+a_{2n}x_n=b_2 \\  
	a_{31}x_1+a_{32}x_2+a_{33}x_3+\cdots+a_{3n}x_n=b_3 \\
	\cdots \\
	a_{m1}x_1+a_{m2}x_2+a_{m3}x_3+\cdots+a_{mn}x_n=b_m
\end{matrix}\right. $$在系数矩阵的右边添上一列,这一列是线性方程组的等号右边的值,构成增广矩阵,记作$\tilde{\mathbf{A} } $

\begin{definition}{增广矩阵(augmented matrix)}
	线性方程组$$\left\{\begin{matrix} 
		a_{11}x_1+a_{12}x_2+a_{13}x_3+\cdots+a_{1n}x_n=b_1 \\  
		a_{21}x_1+a_{22}x_2+a_{23}x_3+\cdots+a_{2n}x_n=b_2 \\  
		a_{31}x_1+a_{32}x_2+a_{33}x_3+\cdots+a_{3n}x_n=b_3 \\
		\cdots \\
		a_{m1}x_1+a_{m2}x_2+a_{m3}x_3+\cdots+a_{mn}x_n=b_m
	\end{matrix}\right. $$其系数构成一个矩阵后在最后一列添加系数矩阵,记作$\tilde{\mathbf{A}}$,即$$\tilde{\mathbf{A}}:=\begin{pmatrix}  
		a_{11}& a_{12}& \cdots  & a_{1n} & b_1 \\  
		a_{21}& a_{22}& \cdots  & a_{2n} & b_2\\  
		\vdots & \vdots & \ddots & \vdots \\  
		a_{m1}& a_{m2}& \cdots  & a_{mn} & b_m
	  \end{pmatrix}  
	  $$
\end{definition}

通过增广矩阵的初等变换,我们可以求解线性方程组,通常这个方法叫做高斯消元法,接下里我们看一个例子:

\begin{example}
	写出线性方程组$$\left\{\begin{matrix} 
		x+y+z=4 \\  
		2x+4y+5z=16 \\
		x+5y-3z=0
	\end{matrix}\right. $$的增广矩阵
	\tcblower
	\textcolor{purple}{\textbf{解}}:见下讲解
\end{example}

我们提取该方程组的增广矩阵,即$$\tilde{A}=\begin{pmatrix}
	1 & 1 & 1 & 4\\
	2 & 4 & 5 & 16\\
	1 & 5 & -3 & 0
\end{pmatrix}$$应用矩阵的初等\textcolor{red}{行}变换就可以求解该线性方程组,下面先讲解一下什么是矩阵的初等变换。

\subsection{矩阵的初等变换}

首先为了说明变换后的矩阵和原矩阵的关系,我们使用``等价''来描述它们,需要注意的是,它们通常并不相等。设矩阵$\mathbf{A}$经过初等变换后变为$\mathbf{A}'$,则$\mathbf{A}$与$\mathbf{A}'$等价,记作$$\mathbf{A}\sim \mathbf{A}'$$如无特殊说明,我们这里都使用初等行变换,实际上对矩阵的列进行变换操作在某些情况下也是等价的。

\subsubsection{交换两行}

交换两行是矩阵的初等变换,例如$$\begin{pmatrix}
	1 & 3 & 3 & 3\\
\textcolor{blue}{2} & \textcolor{blue}{2} & \textcolor{blue}{2} & \textcolor{blue}{1}\\
	\textcolor{red}{4} & \textcolor{red}{2} & \textcolor{red}{1} & \textcolor{red}{3}
\end{pmatrix}\sim\begin{pmatrix}
	1 & 3 & 3 & 3\\
	\textcolor{red}{4} & \textcolor{red}{2} & \textcolor{red}{1} & \textcolor{red}{3}\\
	\textcolor{blue}{2} & \textcolor{blue}{2} & \textcolor{blue}{2} & \textcolor{blue}{1}
\end{pmatrix}$$

\subsection{某一行乘以常数k}

某一行乘以常数$k$是矩阵的初等变换,例如$$\begin{pmatrix}
	1 & 2 & 3 & 1\\
	\textcolor{red}{4} & \textcolor{red}{2} & \textcolor{red}{1} & \textcolor{red}{2}\\
	2 & 2 & 2 & 1
\end{pmatrix}\sim \begin{pmatrix}
	1 & 2 & 3 & 1\\
	\textcolor{red}{4k} & \textcolor{red}{2k} & \textcolor{red}{k} & \textcolor{red}{2k}\\
	2 & 2 & 2 & 1
\end{pmatrix},k\in \mathbb{C}$$

\subsection{某个数乘以某一行并加到另一行中去}

某个数乘以某一行并加到另一行中去是矩阵的初等变换,例如$$\begin{pmatrix}
	\textcolor{red}{1} & \textcolor{red}{2} & \textcolor{red}{3} & \textcolor{red}{4}\\
	1 & 2 & 2 & 4\\
	2 & 3 & 4 & 5\\
	4 & 5 & 6 & 7
\end{pmatrix}\sim \begin{pmatrix}
	\textcolor{red}{1} & \textcolor{red}{2} & \textcolor{red}{3} & \textcolor{red}{4}\\
	1 & 2 & 2 & 4\\
	2+\textcolor{red}{2} & 3+\textcolor{red}{4} & 4+\textcolor{red}{6} & 5+\textcolor{red}{8}\\
	4 & 5 & 6 & 7
\end{pmatrix}$$

\subsubsection{最简矩阵}

矩阵经过一定的初等变换,可化简为最简矩阵,在给出最简矩阵的定义之前,先给出一个术语来描述一个特定行。

\begin{definition}{零行}
	矩阵的某一行均为0,则称这一行为为零行,否则为非零行。
\end{definition}

下面给出行阶梯矩阵与最简矩阵的定义。

\begin{definition}{行阶梯形矩阵与最简矩阵}
	非零矩阵若满足,非零行在零行的上面,且非零行的首个非零元素在列的上一行(如果存在的话)的首个非零元素所在列的后面,例如下面一个就是行阶梯形矩阵\footnote{如红色标出的0元素,像一个阶梯一样从左往右下降}$$\mathbf{A}=\begin{pmatrix}
		1 & 0 & 4 & 1 & 2 & 3\\
		\textcolor{red}{0} & 1 & 3 & 5 & 0 & 3\\
		\textcolor{red}{0} & \textcolor{red}{0} & \textcolor{red}{0} & 2 & 3 & 7\\
		\textcolor{red}{0} & \textcolor{red}{0} & \textcolor{red}{0} & \textcolor{red}{0} & \textcolor{red}{0} & \textcolor{red}{0}
	\end{pmatrix}$$
	特别地,当行阶梯形矩阵的每一行第一个非零元素,例如$$\begin{pmatrix}
		\textcolor{blue}{1} & 0 & 4 & 1 & 2 & 3\\
		\textcolor{red}{0} & \textcolor{blue}{1} & 3 & 5 & 0 & 3\\
		\textcolor{red}{0} & \textcolor{red}{0} & \textcolor{red}{0} & \textcolor{blue}{1} & 3 & 7\\
		\textcolor{red}{0} & \textcolor{red}{0} & \textcolor{red}{0} & \textcolor{red}{0} & \textcolor{red}{0} & \textcolor{red}{0}
	\end{pmatrix}$$我们称这个矩阵为最简矩阵\footnote{最简矩阵一般指的是最简行矩阵,如果是列最简,我们会明确说明是``列最简矩阵''}。
\end{definition}

\begin{corollary}
	任何矩阵经过有限次的行变换都可以转化为最简矩阵。
\end{corollary}

\subsection{高斯消元法}

线性方程组的增广矩阵可以由行变换变为最简行矩阵,通常可用于解无法使用Cramer法则的线性方程组\footnote{斋藤正彦在它的著作《线性代数入门》这本书中,讲到了有位计算机学家告诉他,Cramer法则不适合用于大量行列式的数值计算,相比之下高斯消元法更好},这种方法叫做高斯消元法。

首先看线性方程组有唯一解的情况,即在Cramer法则下系数矩阵的行列式值不为0,下面是一个例子$$\left\{\begin{matrix} 
	x+y+z=4 \\  
	2x+4y+5z=16 \\
	x+5y-3z=0
\end{matrix}\right. $$我们提取该方程组的增广矩阵为$$\tilde{\mathbf{A}}=\begin{pmatrix}
	1 & 1 & 1 & 4\\
	2 & 4 & 5 & 16\\
	1 & 5 & -3 & 0
\end{pmatrix}$$将其变化为行阶梯矩阵,首先将第1行乘以$-2$加到第2行,我们可得等价矩阵$$\begin{pmatrix}
	1 & 1 & 1 & 4\\
	2 & 4 & 5 & 16\\
	1 & 5 & -3 & 0
\end{pmatrix}\sim \begin{pmatrix}
	1 & 1 & 1 & 4\\
	0 & 2 & 3 & 8\\
	1 & 5 & -3 & 0
\end{pmatrix}$$将第1行乘以$-1$加到第3行,我们继续可得等价矩阵$$\begin{pmatrix}
	1 & 1 & 1 & 4\\
	0 & 2 & 3 & 8\\
	1 & 5 & -3 & 0
\end{pmatrix}\sim \begin{pmatrix}
	1 & 1 & 1 & 4\\
	0 & 2 & 3 & 8\\
	0 & 4 & -4 & -4
\end{pmatrix}$$将第3行除以$-2$加到第2行后,交换2,3两行我们继续可得等价矩阵$$\begin{pmatrix}
	1 & 1 & 1 & 4\\
	0 & 2 & 3 & 8\\
	0 & 4 & -4 & -4
\end{pmatrix}\sim \begin{pmatrix}
	1 & 1 & 1 & 4\\
	0 & -2 & 2 & 2\\
	0 & 0 & 5 & 10
\end{pmatrix}$$最后将第2行除以$-2$,第3行除以5可得矩阵$$\begin{pmatrix}
	1 & 1 & 1 & 4\\
	0 & -2 & 2 & 2\\
	0 & 0 & 5 & 10
\end{pmatrix}\sim \begin{pmatrix}
	1 & 1 & 1 & 4\\
	0 & 1 & -1 & -1\\
	0 & 0 & 1 & 2
\end{pmatrix}$$如果我们重写为方程组即$$\left\{\begin{aligned} 
	x+y+z&=4 \\  
	y-z&=-1\\
	z&=2
\end{aligned}\right. $$

将结果逐个带入线性方程组,可得$x=1,y=1,z=2$。下面我们给出线性方程组有无穷个解的情况。

\begin{example}
	使用高斯消元法求解线性方程组$$\left\{\begin{matrix} 
	x+2y+4z=2 \\  
	2x+y-z=1\\
	x+y+z=1
\end{matrix}\right. $$
\tcblower
\textcolor{purple}{\textbf{解}}:$x=3z,y=1-3z,z=z$见下讲解
\end{example}

我们提取该方程组的增广矩阵为$$\tilde{\mathbf{A}}=\begin{pmatrix}
	1 & 2 & 4 & 2\\
	2 & 1 & -1 & 1\\
	1 & 1 & 1 & 1
\end{pmatrix}$$首先讲第3行乘以3,我们可得$$\begin{pmatrix}
	1 & 2 & 4 & 2\\
	2 & 1 & -1 & 1\\
	1 & 1 & 1 & 1
   \end{pmatrix}\sim \begin{pmatrix}
	1 & 2 & 4 & 2\\
	2 & 1 & -1 & 1\\
	3 & 3 & 3 & 3
\end{pmatrix}$$