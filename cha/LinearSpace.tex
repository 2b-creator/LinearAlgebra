\chapter{线性空间}
\begin{center}
	\textcolor[RGB]{255, 0, 0}{\faHeart}越想贴近事实,不明白的事情就越多.\textcolor[RGB]{255, 0, 0}{\faHeart}
\end{center}
\rightline{——《宝石之国》}
\vspace{-5pt}
\begin{center}
	\pgfornament[width=0.36\linewidth,color=lsp]{88}
\end{center}

\section{标量与向量}

\subsection{数的运算}

从小学一年级开始我们就接触了整数,并且学会进行加减乘除;长大一些后,我们了解到了分数,将数系扩充到了有理数部分;再后来初中阶段我们遇到了平方根这种运算,接触到了无理数,我们把这些数都统称为实数,后来我们又了解了 $-1$ 也能被开平方根,开创了复数领域;同时在高中阶段,我们也学习了简单的集合论,通常我们会用几个记号来表示这些数域。

\begin{definition}{数集的符号}
	这些数集有着相对应的符号:
	\begin{itemize}
		\item $\mathbb{N}$ 表示所有自然数的集合,其元素是所有非负整数。
		\item $\mathbb{Z}$ 表示所有整数的集合。
		\item $\mathbb{Q}$ 表示所有有理数的集合,其中有理数被定义为所有能够表示为 $\frac{m}{n},m\in \mathbb{Z},n\in \mathbb{Z}$ 的数。
		\item $\mathbb{R}$ 表示所有实数的集合。
		\item $\mathbb{C}$ 表示所有复数的集合。
	\end{itemize}
\end{definition}
%此外,德国数学家Frobenius也证明,在当下的数学体系中 $\mathbb{C}$ 是最大的数域,

由于这门课不是实分析或高等代数,所以不会详细讲解这些数集怎么构造,读者只需要对此有一个简单的认识就可以了。众所周知,数的运算分别有加减乘除四种,如果读者觉得自己小学数学不错的话,一定知道下面的定理。

\begin{axiom}{域公理}
	对于这些数的基本算数性质有:
	\begin{enumerate}
		\item \textbf{交换律(communitativity)}:对于所有的$\alpha,\beta \in \mathbb{C}$有$\alpha+\beta=\beta+\alpha,\alpha\beta=\beta\alpha$;
		\item \textbf{结合律(associativity)}:对于所有的$\alpha,\beta,\gamma \in \mathbb{C}$有$(\alpha+\beta)+\gamma=\alpha+(\beta+\gamma),(\alpha\beta)\gamma=\alpha(\beta\gamma)$;
		\item \textbf{单位元(identities)}:对于所有的$\lambda \in \mathbb{C}$都有$\lambda+0=\lambda,1\lambda=\lambda$;
		\item \textbf{加法逆元(additive inverse)}:对于所有的$\alpha \in \mathbb{C}$都存在唯一的$\beta \in \mathbb{C}$使得$\alpha+\beta=0$;
		\item \textbf{乘法逆元(multiplicative inverse)}:对于每个$\alpha \in\mathbb{C}$且$\alpha\neq 0$,都存在唯一的$\beta \in \mathbb{C}$使得$\alpha\beta=1$;
		\item \textbf{分配律(distributive)}:对于所有的$\alpha,\beta,\gamma \in \mathbb{C}$都有对于所有的$\alpha(\beta+\gamma)=\alpha\beta +\alpha\gamma$.
	\end{enumerate}
\end{axiom}

如果一个数域$\mathrm{S}$和$\mathbb{C}$一样满足上述性质,我们称$\mathrm{S}$中的元素为标量(scalar),当然$\mathbb{C}$和$\mathbb{R}$的所有元素也是标量。

\subsection{标量}

如上所述,标量通常强调的是,这一个元素是一个数,而不是向量(稍后给出定义),实际上只是``数''的一种花里胡哨的叫法而已。德国数学家Frobenius 已经证明,当下数学体系内再也没有比$\mathbb{C}$更大的数系了,所以我们可以称对于每一个$\alpha \in \mathbb{C}$,$\alpha$是标量,例如 $3+4\mathrm{i}$ 是一个标量。

\begin{example}
	前文所述,我们说过复数是最大的数系,请求出$\sqrt{\mathrm{i}}$的两个解,其中$\mathrm{i}=\sqrt{-1}$。
	\tcblower
	\textcolor{purple}{\textbf{解}}:设$\sqrt{\mathrm{i}}=z=(a+b\mathrm{i})$,其中$a,b \in \mathbb{R}$,通过两边将其平方可得$$\mathrm{i}=(a+b\mathrm{i})^2=a^2-b^2+2ab\mathrm{i}$$观察式子联立方程组为$$\left\{\begin{matrix} 
		a^2-b^2=0 \\  
		2ab=1
	\end{matrix}\right. $$
	解此方程组可得$$\left\{\begin{matrix} 
		a=b \\ 
		a=\pm \frac{\sqrt{2} }{2} 
	\end{matrix}\right. $$
	所以$\sqrt{\mathrm{i}}$的两个解为$$z_1=\frac{\sqrt{2}}{2}+\frac{\sqrt{2}}{2}\mathrm{i},z_2=-\frac{\sqrt{2}}{2}-\frac{\sqrt{2}}{2}\mathrm{i}$$
\end{example}

\subsection{元组}

若$n\in \mathbb{N}$,我们将由$n$个数按顺序排列成为的一个组称之为元组(tuple),通常的记法为:有小括号包含这$n$个数,且有序排列,中间由逗号隔开,例如$(a,b)$是一个长度为2的元组,如果长度为$n$且由$z_1,z_2,z_3,\cdots,z_n$这些数组成的元组的记法可能如下所示:$\left( z_1,z_2,z_3,\cdots,z_n\right) $。

\begin{ascolorbox1}{思考}
	集合是元组吗?
\end{ascolorbox1}

对于这个思考题,很显然答案是不对的,因为元组强调以下两点:
\begin{itemize}
	\item 有序性,$(3,5)$和$(5,3)$并不相等,而$\{3,5\}$和$\{5,3\}$是相等的集合;
	\item 可重复,$(3,3)$和$(3,3,3)$并不相等,而$\{3,3\}$和$\{3,3,3\}$是相等的集合,它们等同于$\{3\}$。
\end{itemize}

\subsection{笛卡尔积}

接下来我们慢慢引入向量的概念,在讲向量之前,笔者想和大家谈一谈如何生成这样的空间来容纳我们向量的概念,这里有一个定义叫做笛卡尔积(Cartesian product)。

\begin{definition}{笛卡尔积}
	两个集合${\displaystyle X}$和${\displaystyle Y}$的笛卡尔积是所有可能的有序对组成的集合,其中有序对的第一个对象是$X$的成员,第二个对象是$Y$的成员,记作:$${\displaystyle X\times Y := \footnote{符号$:=$表示定义为,用来明确地给某个符号或概念赋予一个新的定义,例如:妈妈$:=$养育与教养子女成长的女性} \left\{\left(x,y\right)\mid x\in X \text{且} y\in Y\right\}}$$
\end{definition}

举个很常见的例子,如果集合$X$是13个元素的点数集合${\displaystyle \left\{A,K,Q,J,10,9,8,7,6,5,4,3,2\right\}}$而集合$Y$是四种花色$\{\spadesuit, \heartsuit, \diamondsuit, \clubsuit\}$,则这两个集合的笛卡儿积是有52个元素的标准扑克牌的集合:$$X\times Y = \left\lbrace (A,\spadesuit),(K,\spadesuit),(Q,\spadesuit)\cdots (2,\spadesuit),\cdots,(3,\clubsuit),(2,\clubsuit)\right\rbrace $$

\subsection{空间 $\mathrm{S}^n$}

我们定义$\mathbb{R}^2=\mathbb{R} \times \mathbb{R}$,根据笛卡尔积的运算规则,集合$\mathbb{R}^2$为所有有序实数2元组构成,即$$\mathbb{R}^2=\{(x,y)\mid x,y \in \mathbb{R}\}$$在现实生活中,我们可以将它看成是一个二维平面,而$\mathbb{R}^3$在现实生活中,我们可以将它看成是一个三维空间。

我们上述提到集合$\mathrm{S}$内的元素是一个标量,那么$\mathrm{S}^n$则由无数个的$n$元组构成,例如$\mathbb{C}^6$就是含有6个有序元素构成的一个空间,虽然当$n>3$的时候在实际生活中想象它们绝非易事,但是数学就是一门由具体到抽象的科目,我们可以使用类比的手法,让它们和在$\mathbb{R}^2$与$\mathbb{R}^3$上运算一样有自己的实际意义。

\subsection{向量}

如之前所述,我们终于提到了向量(vector)这个概念,只不过可能会令读者失望的是,数学科目上的向量是一个非常抽象的东西,那么我们把目光转换到另外一门和数学紧密相关的科目上:物理。

学基础物理的学生总是很实际,他们将数学转化为解决现实生活中问题的工具,所以在他们眼中,向量一般是一个箭头,它具有方向和大小,且这两个特征不会随着向量的移动发生变化,因此该形式的向量可以在空间中的任何一个位置。

所以我们在此非严格定义一个$n$元组被看作箭头时,就是一个$n$维向量(稍后我们会给出它的更准确的定义);下面是一个二维向量在平面$\mathbb{R}^2$内

\begin{figure}[htbp]    % 常规操作\begin{figure}开头说明插入图片
		% 后面跟着的[htbp]是图片在文档中放置的位置, 也称为浮动体的位置, 关于这个我们后面的文章会聊聊, 现在不管, 照写就是了
		\centering            % 前面说过, 图片放置在中间
		\subfloat[一个指向点$(2,2)$的向量]   % 第一张子图的下标(注意: 注释要写在[]中括号内)
		{
			\label{fig:a.vector}\includegraphics[width=0.5\textwidth]{eps/VectorInR2.eps}
			% \label{}命令为每个子图添加标签, 方便在正文中引用. 如果你不需要引用的话, 也可以不加这个命令, 写法在下面有: 
			% \label{}命令的{}内第一个{}中的内容fig:subfig1就是你插入的这张子图的标签, 注意每个标签都不能一样, 要用合适的编号去区分, 比如1, 2, 3......
			% \label{}命令中{}内\includegraphics[]{}就是真正插入图片的命令, []中的是图片的一些参数, {}就是图片的相对路径
			% width=0.4\textwidth 就是设置图片的大小, 这里设置的是文档宽度(\textwidth)的0.4倍, 在设置时注意不要超宽, 不然会报错, 大家多设置几个数尝试一下就能理解了
		}
		\subfloat[这两个向量是同一个,因为它们的大小和方向都一样]
		{
			\label{fig:same.vector}\includegraphics[width=0.5\textwidth]{eps/SameVectorInR2.eps}
		}
		\caption{二维向量}    % 整个图片的说明, 注释写在{}内
		\label{fig:intro.vector}            % 整个图片的标签编号, 注意这里跟子图是一样的道理, 标签不能重复 
\end{figure}

\section{向量的运算法则}

\subsection{加法}

\begin{definition}{向量的加法}
	若$x,y\in \mathrm{S}^n,x=(x_1,x_2,x_3,\cdots,x_n),y=(y_1,y_2,y_3,\cdots,y_n)$则$x+y$定义为$$x+y=(x_1+y_1,x_2+y_2,x_3+y_3,\cdots x_n+y_n)$$
\end{definition}

实际上两个向量相加,就是将它们元组里面的所有元素依次相加得到一个新的元组。

\begin{ascolorbox1}{思考}
	维度不同的向量是否可以相加?
\end{ascolorbox1}

要解答这个问题,我们需要返回到定义中,我们只定义了维度相同都为$n$的向量相加,并没有定义维度不同的向量相加的情况,所以这是一个未定义的行为;所以下面的对向量的运算均是在同一维度下进行。同理,向量与标量无法相加。

同样,向量的加法也满足交换律:
\begin{corollary}
	若$x,y\in \mathrm{S}^n$,则$x+y=y+x$。
\end{corollary}

下面是简单的对此性质的证明(读者不作要求)。

\begin{proof}
	若$x,y\in \mathrm{S}^n,x=(x_1,x_2,x_3,\cdots,x_n),y=(y_1,y_2,y_3,\cdots,y_n)$则$$x+y=(x_1+y_1,x_2+y_2,x_3+y_3,\cdots x_n+y_n)$$根据标量加法的交换律可知
	\begin{equation*}
	x+y=(y_1+x_1,y_2+x_2,y_3+x_3,\cdots y_n+x_n)=y+x \teoe
	\end{equation*}
\end{proof}

如果我们将其反映在二维坐标系上,那应该是这样的
\begin{figure}[htbp]
	\centering
	\includegraphics[width=0.7\linewidth]{eps/SumOf2Vectors}
	\caption{平面内两向量之和}
	\label{fig:sumof2vectors}
\end{figure}


\subsection{0 向量}

我们之前说过,向量与标量无法相加,所以在$\mathrm{S}^n$内我们定义$n$维0向量作为加法的单位元,即若$x,y\in \mathrm{S}^n$,$x+y=x$,那么$y$就是0 向量。

\begin{definition}{0 向量}
	若$\alpha,\beta \in \mathrm{S}^n$且$\alpha+\beta = \alpha$,那么$\beta$为空间$\mathrm{S}^n$的单位元,我们将其称为0向量,显然0向量就是一个全是由0组成的$n$元组,记作$\boldsymbol{0}$\footnote{实际上是以粗体形式出现的0},那么$$\boldsymbol{0}=(0,0,0,\cdots,0)$$
\end{definition}

\subsection{加法逆元}

类似于标量加法的$x,y\in \mathbb{C},x+y=0$,其中标量0表示标量加法的单位元,其中$y=-x$,那么在空间$\mathrm{S}^n$中同样表示若$\alpha,\beta \in \mathrm{S}^n,\alpha+\beta=\boldsymbol{0}$,$\beta=-\alpha$表示$\alpha$的加法逆元。

\begin{definition}{加法逆元}
	对于$\alpha \in \mathrm{S}^n$,$\alpha$的加法逆元表示满足下面条件的向量$-\alpha\in \mathrm{S}^n$有$$\alpha+(-\alpha)=\boldsymbol{0}$$换而言之当$\alpha=(a,b,c,d,\cdots)$时,$-\alpha=(-a,-b,-c,-d,\cdots)$
\end{definition}

\subsection{标量乘法}

\begin{definition}{向量的标量乘法}
	若标量$\lambda \in \mathrm{S}$,其对一个向量$x=(x_1,x_2,x_3,\cdots,x_n)$的乘积满足以下的运算法则$$\lambda(x_1,x_2,x_3,\cdots,x_n)=(\lambda x_1,\lambda x_2,\lambda x_3,\cdots,\lambda x_n)$$
\end{definition}

实际上如图\ref{fig:mulfsn}所示,向量的标样乘法从$\mathbb{R}^2$的几何意义来看表示的是向量长度的缩放倍率。例如$x=(1,2)$那么$2x=(2,4)$.

\begin{figure}[htbp]
	\centering
	\includegraphics[width=0.7\linewidth]{figure/eps/MulFSn}
	\caption{向量长度的缩放倍率}
	\label{fig:mulfsn}
\end{figure}

同时,向量的标量乘法满足标量结合律。
\begin{corollary}
	若$a,b\in \mathrm{S},x\in \mathrm{S}^n$,满足$(ab)x=a(bx)$。
\end{corollary}

至此,我们对向量的运算法则做一个总结,我们目前只定义了2种运算法则,一种是向量和,另一种是标量积,下面我们会用这两种运算讲解一个特殊的集合,使得这个集合由这两部分构成。

\begin{example}
	已知$x\in \mathbb{R}^4$且满足$(1,1,4,-5)+2x=(5,1,-4,1)$,求$x$。
	\tcblower
	\textcolor{purple}{\textbf{解}}:等式两边同时加上$(1,1,4,-5)$的逆元$(-1,-1,-4,5)$可得$$2x+(0,0,0,0)=(5,1,-4,1)+(-1,-1,-4,5)=(4,0-8,6)$$设$x=(a,b,c,d)$可得$$(2a,2b,2c,2d)=(4,0,-8,6)$$由此可以得到方程组$$\left\{\begin{matrix} 
		2a=4 \\
		2b=0 \\
		2c=-8 \\
		2d=6
	\end{matrix}\right. $$所以可得$x=(2,0,-4,3)$
\end{example}

读者平时在作业的时候无需大费周章地严格用定理来写题目,如果你熟练的话可以直接自己写一个关于向量的减法法则(实际上就是加法法则的逆运算)来做题。

\section{线性空间$\mathbb{R}^2$与$\mathbb{R}^3$}
\subsection{线性空间}

如果要刻画线性空间,我们先从空间说起。

前文我们提到了``空间$\mathrm{S}^n$'',那么我们可以在此断章取义地说,空间就是一个集合,而线性空间是一个特殊的集合,我们给它一个符号$V$来表示。

下面给出定义来描述线性空间特殊的地方,线性空间$V$是一个带有加法和标量乘法的集合,下面我们将前面的向量的运算法则抽象化来定义它们。不过在此之前,我希望和大家来聊一聊一些你可能不是很了解的名词,例如函数(function)。
\begin{definition}{函数(一元)}
	对于任意的非空集合$A,B$,对集合$A$施加一个法则$f$使得$A$中的元素$x$可以转变为集合$B$中的元素$y$,记作$f(x)$,即$y=f(x)$。
\end{definition}

上面讲述了函数的一般情形,那就是一元函数;一元函数有一个显著的特点,就是``输入一个元素并输出一个元素'',当然还有多元函数,下面请读者思考。

\begin{ascolorbox1}{思考}
	两个可以相加的元素进行加法运算,这是一种函数吗?
\end{ascolorbox1}

回答是肯定的,只不过这就不是一元函数了,这是二元函数,即``输入两个元素并输出一个元素'',翻译成数学的符号语言可以是:若$a,b\in A$则$f(a,b):=a+b$。同理,如果我们将集合内的元素特殊化为一种向量,那就是若$a,b\in \mathrm{S}^n$则$f(a,b):=a+b$。

那我们回到线性空间上来,如果一个集合$V$满足下面这一些条件的我们称其为线性空间\footnote{有些教材会叫做向量空间}。

\begin{definition}{线性空间上的加法与标量乘法}
	若集合$V$与数集$\mathrm{S}$满足以下的性质,则称$V$为$\mathrm{S}$上的线性空间:
	\begin{enumerate}
		\item $V$上的加法是一个函数使得任意一对$u,v \in V$经过法则$f(u,v)=u+v$转变为$V$中的另一个元素,这说明了$u+v\in V$。
		\item $V$上的标量乘法是一个函数使得任意一个$\alpha \in \mathrm{S}$和$u\in V$经过法则$f(u)=\alpha u$转变为$V$中的另一个元素,这说明了$\alpha u \in V$。
	\end{enumerate}
\end{definition}

根据的定义,线性空间$V$具有如下的性质:

\begin{definition}{线性空间的性质}
	\label{linear.space.sit}
	线性空间就是带有加法和标量乘法的集合$V$,满足如下的性质:
	\begin{itemize}
		\item 交换性(commutativity):对于所有的$u,v\in V$都有$u+v=v+u$;
		\item 结合性(associativity):对于所有的$u,v,w \in V$和$a,b\in \mathrm{S}$都有$(u+v)+w=u+(v+w)$和$(ab)v=a(bv)$;
		\item 加法单位元(additive identity):存在$\boldsymbol{0}\in V$使得对于所有的$v\in V$都有$v+\boldsymbol{0}=v$;
		\item 加法逆元(additive inverse):对于每一个$v\in V$都存在$w\in V$使得$v+w=\boldsymbol{0}$;
		\item 乘法单位元(multiplicative identity):对于所有的$v\in V$都存在$w\in V$使得$v+w=0$;
		\item 分配性质(distributive properties):对于所有的$a,b \in \mathrm{S}$和$u,v \in V$都有$a(u+v)=au+av$和$(a+b)v=av+bv$。
	\end{itemize}
\end{definition}

\begin{ascolorbox1}{思考}
	空集可以是线性空间吗?
\end{ascolorbox1}

答案为不是,因为空集$A$它不存在单位元$\boldsymbol{0}\in A$所以并不能是线性空间,此外,需要注意的是我们一般会说$V$是$\mathrm{S}$上的线性空间而不会直接说$V$是线性空间,因为我们在定义线性空间的时候,依赖了$\mathrm{S}$这个标量数集。

如之前所承诺的,下面给出向量的定义。

\begin{definition}{向量}
	线性空间中的元素称为向量。
\end{definition}

读者可能会觉得这样给出会有些突然,我们不妨反过来看看向量作为线性空间$V$的元素,是否满足定义\ref{linear.space.sit}的6个条件,然后与读者高中所学的向量的特性对比,发现是否吻合。如果读者觉得这样做还是太抽象,我们不妨研究一下在线性空间$\mathbb{R}^2$与$\mathbb{R}^3$中向量的相关图形表现。

\subsection{直线,平面与三维空间上的向量}

向量同时具有长度和方向两种特性,如果我们在$\mathbb{R}^2$上绘制平面直角坐标系来图像化向量这一个概念的话,向量就是一个有向箭头,由起点和终点构成,若一个有向箭头的起点为$P$,终点为$Q$,我们将此有向箭头\footnote{向量与有向箭头的区别在于,长度相同且方向相同的有向箭头,可以有很多个表示,如图\ref{fig:same.vector}所示,而这两个有向箭头表示的向量是同一个。不过,通常而言,该有向线段也可直接称为向量}记为$\overrightarrow{PQ}$。若$a\in \mathbb{R}^2,a=\overrightarrow{PQ}$,则向量$a$的长度为线段$PQ$的长度,记作$\left \| a \right \| $。

\begin{definition}{$\mathbb{R}^2$上向量的长度}
	若$x\in \mathbb{R}^2$且$x=(x_1,x_2)$,则$x$的长度$\left \| x \right \| $为$$\left \| x \right \| := \sqrt{x_1^2+x_2^2}$$
\end{definition}

\begin{example}
	已知$a,b\in \mathbb{R}^2$,点$O$为平面直角坐标系的原点,$P_1,P_2$分别为一条线段的两个端点,$M$为该线段的重点,其中$a=\overrightarrow{OP_1},b=\overrightarrow{OP_2}$,求$\overrightarrow{OM}$。
	\tcblower
	\textcolor{purple}{\textbf{解}}:在长度上$\left \| \overrightarrow{P_1M} \right \|=\left \| \overrightarrow{MP_2} \right \|$且在同一条直线上,所以$\overrightarrow{P_1M}=\overrightarrow{MP_2}$,根据向量的加法法则有$$\overrightarrow{OM}=\overrightarrow{OP_1}+\overrightarrow{P_1M},\overrightarrow{OM}=\overrightarrow{OP_2}-\overrightarrow{P_1M}$$前后两式相加可得$$\overrightarrow{OM}=\frac{a+b}{2}$$
\end{example}

此外由于向量具有方向性,且可以随处移动,由此我们也可以由此性质确定$\mathbb{R}^2$和$\mathbb{R}^3$空间中直线的方向并确定直线的方程。

\begin{definition}{直线的方向向量}
	空间直线的方向用一个与该直线平行的有向箭头表示,该有向箭头所表示的向量即为空间直线的方向向量。
\end{definition}

\begin{figure}[htbp]
	\centering
	\includegraphics[width=0.7\linewidth]{figure/eps/LineAndVector2d}
	\caption{方向向量与直线方程}
	\label{fig:lineandvector2d}
\end{figure}

如图\ref{fig:lineandvector2d}所示,$P_1$和$P_2$都在直线$l$上,设$x,x_1\in \mathbb{R}^2,x=\overrightarrow{OP},x_1=\overrightarrow{OP_1}$,有向箭头$\overrightarrow{P_1P_2}$在直线$l$上,若向量$a\in \mathbb{R}^2$满足$a=\overrightarrow{P_1P_2}$,则该向量$a$是这条直线的方向向量。

既然知道了直线的方向向量和$O$到直线上某一点的有向箭头所表示的向量,我们就可以建立一个等式来唯一确定一条直线,设$t\in \mathbb{R}$,那么对于任意处于直线上的$P$点,总存在$t$使得$x=\overrightarrow{OP}$满足$$x=x_1+ta$$如果我们设直线上的$P$点坐标为$(x,y)$那么$x=(x,y)$设$P_1$的坐标是$(x_1,y_1)$,$P_2$的坐标是$(x_2,y_2)$,于是乎$a=(x_2-x_1,y_2-y_1)$,由$x=x_1+ta$并代入坐标,根据向量的加法与标量乘积的运算定理,我们可以得到下面的式子$$\left\{\begin{matrix} 
	x-x_1=t(x_2-x_1) \\  
	y-y_1=t(y_2-y_1)
\end{matrix}\right. $$所以我们可以得到这就是经过$P_1$和$P_2$两点的参数方程,如果要进一步往下化简的话,我们可以得到$$\frac{x-x_1}{x_2-x_1}=\frac{y-y_1}{y_2-y_1}$$

\begin{example}
	求二维空间$\mathbb{R}^2$内直线方程$3x+4y-5=0$的一个方向向量。
	\tcblower
	\textcolor{purple}{\textbf{解}}:设$P_1,P_2$点坐标分别为$(x_1,y_1),(x_2,y_2)$,则$a\in \mathbb{R}^2,a=\overrightarrow{P_1P_2}$为其一个方向向量,令$x_1=0$则$\displaystyle y_1=\frac{5}{4}$,$\displaystyle \overrightarrow{OP_1}=\left( 0,\frac{5}{4} \right)$,令$y_2=0$则$\displaystyle x_2=\frac{5}{3}$,$\displaystyle \overrightarrow{OP_2}=\left(\frac{5}{3},0\right)$,$\overrightarrow{P_1P_2}=\overrightarrow{OP_2}-\overrightarrow{OP_1}$所以该直线的一个方向向量为$\displaystyle \left( \frac{5}{3},-\frac{5}{4} \right) $
\end{example}

接下来我们将维度扩展到三维空间$\mathbb{R}^3$内,依然有$$x=x_1+ta$$由此如果$P_1$与$P_2$的坐标分别为$(x_1,y_1,z_1)$和$(x_2,y_2,z_2)$,我们将会得到下面的等式$$\frac{x-x_1}{x_2-x_1}=\frac{y-y_1}{y_2-y_1}=\frac{z-z_1}{z_2-z_1}$$这就是空间内的直线方程(实际上就是两个一次方程)

\begin{example}
	求三维空间$\mathbb{R}^3$内直线方程$$\left\{\begin{matrix} 
		x+y+z=3 \\  
		x+2y+3z=6
	\end{matrix}\right. $$的一个方向向量。
	\tcblower
	\textcolor{purple}{\textbf{解}}:设$P_1,P_2$点坐标分别为$(x_1,y_1,z_1),(x_2,y_2,z_2)$,则$a\in \mathbb{R}^3,a=\overrightarrow{P_1P_2}$为其一个方向向量,令$x_1=1$则$\displaystyle y_1=1,z_1=1$,$\displaystyle \overrightarrow{OP_1}=\left( 1,1,1 \right)$,令$x_2=0$则$\displaystyle y_2=3,z_2=0$,$\displaystyle \overrightarrow{OP_2}=\left(0,3,0\right)$,$\overrightarrow{P_1P_2}=\overrightarrow{OP_2}-\overrightarrow{OP_1}$所以该直线的一个方向向量为$\displaystyle \left( -1,2,-1 \right) $
\end{example}