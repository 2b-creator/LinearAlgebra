
\frontmatter
\thispagestyle{empty}
\newpage
\begin{center}
	\textbf{\LARGE 序言:从等待走向默契}
\end{center}

\vspace{2em}

这本书推荐使用电子介质阅读,或使用彩印书籍以获得最佳观感。

感谢你翻开这本书!

当年我的数学老师告诉我说,书籍可是作者的亲生儿子,你需要花费很多很多时间和精力去创作它,而不是脑子一热说我想写一本书。But,这本书确实是脑子一热诞生的一个项目,只不过开坑一张嘴,填坑跑断腿,不知不觉间这本书已经陪我走过半年时间。出这部书出自于偶然,并且承载了一份很长的等待,从立项开始,内心可以算是挣扎着是否要继续写下去;恍惚间感觉一直都在强加给这件事一个意义。

所以意义我认为算是没有等到,不过等到了一个机会,那就是理科迷的公益课邀请我与大家来聊聊线性代数的意义与价值。如果大家已经接触过线性代数,或许会有和我当年一样学习线性代数时的困惑:

\vspace{1em}

\begin{enumerate}
	\item 上来就是行列式,它们如何得来,为何这么计算?
	\item 什么是矩阵,它们的相加,相乘都有什么意义?
	\item 为什么要引入线性代数这门数学分支,它能够做什么?
\end{enumerate}

\vspace{1em}

经过我们本科的线性代数的学习,我们老师告诉我们线性代数(工科\footnote{有区别于高等代数,只涉及计算技巧,不涉及严格证明})的本质围绕着``解方程''\footnote{如果这本书没有严格说明,那么都是描述解线性方程组}展开,行列式与矩阵不过是解决这类问题的工具,当我们尝试追问,行列式为什么这么计算的时候,数学老师告诉我们,这就像$1+1=2$一样,是一个规定,如果你尝试严格证明,那牵扯的东西会很多。事实上的确如此,不过这本书打算以一个抽象的方式来描述``解方程'',实际上解方程也不过是一个从$\mathbb{R}^n\rightarrow \mathbb{R}^m$的线性映射;那么除了解方程,线性代数也有很多其他意义,例如对空间中的直线与平面作解析几何,对多组数据的线性处理研究等。

在学习线性代数之前,笔者阅读了国内的部分教材,其中包括同济本(人民邮电出版社\footnote{同济大学数学系编著的共有2个出版社,其中最常用的是高等教育出版社,此外人民邮电出版社内容编排更合理一些,但是更为小众}),很经典地,上来就是行列式,缺少动机也缺少直观,不过好在我认为我们线性代数的老师水平不错,能够让大家在不知其所以然的时候知道行列式是什么东西,让我们放下思想包袱,不要去深究这里面的内容;不过数学学习主打一个``兴趣'',如果你深入研究你会发现数学其实还是很好玩的,在不同的领域中数学的应用,或许会让你产生闭环的感觉,发现原来这里也用到了线性代数的知识,就像一位故友一般。所以我很赞同我们大学数学老师的话,那就是学习重在一个过程而不是结果。有人曾言,若离开考试,语文就是一个绝美的化身,那么数学则是充满了理性之美的艺术。

所以写这本书的动机就是希望它是笔者亲自打造的艺术品,希望传达数学的艺术,不仅限于书本与考试的一片三亩方地,而是能够实现自我探索,发现更多知识的天空。当然我们目前学习也是为了考试,所以在后面笔者也会放上大部分课上的笔记与注意事项,具体的书目会书写大概3个部分:

\vspace{1em}

\begin{enumerate}
	\item 理论部分,这个部分会放在第一个讲,从数与集合开始,并产生线性代数的一个研究的容器,逐步生成线性空间音容线性映射的概念。这一部分会有些难度,如果看不明白可以跳过,至于为什么放在第一个,是因为我认为这是线性代数的本质,能够帮助读者对行列式和矩阵有一个不同于``解方程''的认识。该部分包括线性空间,线性映射,内积空间。
	\item 基础部分,讲解行列式,矩阵的计算,依照线性方程组的特性与理论部分辅助理解内容。同时会结合教师讲义讲解计算技巧,满足考试的需要,让这一本书更贴合实际。
	\item 附加部分,讲解多项式,度量空间
    %以及线性泛函上的微积分,
    等,给希望拓展线性代数学习的读者学习交流。
\end{enumerate}

\vspace{1em}

每个包含若干个章节,章节后面会有两组练习题,对于这两组练习题的难度,个人认为要求如下:

\vspace{1em}

\begin{itemize}
	\item A 组是基础的题目,需要读者熟练掌握并了解其含义,不会涉及很难的证明题,是对该章节最基本的认知;几乎会以例题为标准举一反三,所以与该书的例题难度不相上下;
	\item B 组是拓展题,对工科学生来说不应要求掌握,但是可以给想挑战自我的读者一个发挥的空间;
	\item 所有题目的答案将会在本人的 GitHub 主页手写发布。
\end{itemize}

\vspace{1em}

我希望这本书能够帮助所有初次接触线性代数的同学们一些参考,承载等待的时候,多了份默契,这份默契不需要过多的彩排,源于大家的支持,若能够帮助到大家的学习,我何其有幸!

最后感谢在前行路上给予我指导的老师与同学们,特别鸣谢:广西民族大学向子凤同学,南京大学余沛宸同学,江苏理工学院蒋盈峰教师,清华大学李思教授,以及江苏省江阴高级中学翟国华教师;特别致谢江苏理工学院 2023-2024 经济统计学班的全体学生和老师。 


\vspace{2em}

\begin{flushright}
	匡睿同学\\
	July 1, 2025 于长广溪畔
\end{flushright}


%\newpage
%\begin{center}
%	\textbf{\LARGE 内容概述}
%\end{center}
%\begin{ascolorbox17}{内容概述}
%\begin{minipage}[b]{0.49\textwidth}
%	\begin{dinglist}{118}
%		\item 
%%		\item 001 网络爬虫初始
%%		\item 002 浏览器开发者工具使用
%%		\item 003 HTTP协议与HTTPS协议
%%		\item 004 Requests库基本使用
%%		\item 005 xpath与lxml模块使用
%%		\item 006 正则表达式与re模块使用
%%		\item 007 CSS选择器与BS4库使用
%%		\item 008 jquery与PyQuery模块使用
%%		\item 009 Json数据与Json模块的使用
%%		\item 010 数据存储Pandas使用
%	\end{dinglist}
%\end{minipage}
%\begin{minipage}[b]{0.49\textwidth}
%	\begin{dinglist}{118}
%		\item 
%%		\item 011 数据存储MySQL与MongoDb使用
%%		\item 012 爬虫进阶Selenium的使用
%%		\item 013 爬虫进阶多进程爬虫
%%		\item 014 爬虫进阶多线程爬虫
%%		\item 015 爬虫进阶多协程爬虫
%%		\item 016 爬虫进阶异步爬虫
%%		\item 017 爬虫进阶Scapy爬虫
%%		\item 018 爬虫进阶分布式爬虫
%%		\item 019 爬虫进阶APP爬虫
%%		\item 020 爬虫序章反爬
%	\end{dinglist}
%\end{minipage}
%\end{ascolorbox17}
\frontmatter