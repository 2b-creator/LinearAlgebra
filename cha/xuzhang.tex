
\frontmatter
\thispagestyle{empty}
\newpage
\begin{center}
	\textbf{\LARGE 序言:讲述风格决定听者兴趣}
\end{center}

\vspace{2em}
后面再写(
%感谢你翻开这本书!
%
%既然下定决心要写书,那么思考受众群体肯定是头等事务。作为一门对于大多数偏理工科的学生来说必修的科目,在你第一次接触线性代数,你可能会很想知道行列式、矩阵是如何得来的。而市面上大部分高等教育的教材书籍只会讲解它们怎么算,当然,这不代表如何算不重要,而是缺少对线性代数有一个直观的引入,所以这本书的面向对象是第一次接触线性代数的同学们。
%
%受到国外教材的启发,特别是普林斯顿lifesaver读本的三剑客,让我对讲述风格产生了极大的兴趣。说简单点,就是\textbf{说人话},所以本书的副标题写的是``与我的思考'',所以这本书更偏向于是一种笔记而非教材,希望我学习的心路历程能够帮助到大家能够更好地了解线性代数。此外普林斯顿读本以幽默风趣的形式,为我们娓娓道来相关知识,很大程度上为读者建立了一个很好的兴趣,俗话说,兴趣是最好的老师,笔者认为只有真正拥有兴趣,才能学好这一门科目。



\vspace{2em}

\begin{flushright}
	匡睿同学\\
	\today
\end{flushright}


%\newpage
%\begin{center}
%	\textbf{\LARGE 内容概述}
%\end{center}
%\begin{ascolorbox17}{内容概述}
%\begin{minipage}[b]{0.49\textwidth}
%	\begin{dinglist}{118}
%		\item 
%%		\item 001 网络爬虫初始
%%		\item 002 浏览器开发者工具使用
%%		\item 003 HTTP协议与HTTPS协议
%%		\item 004 Requests库基本使用
%%		\item 005 xpath与lxml模块使用
%%		\item 006 正则表达式与re模块使用
%%		\item 007 CSS选择器与BS4库使用
%%		\item 008 jquery与PyQuery模块使用
%%		\item 009 Json数据与Json模块的使用
%%		\item 010 数据存储Pandas使用
%	\end{dinglist}
%\end{minipage}
%\begin{minipage}[b]{0.49\textwidth}
%	\begin{dinglist}{118}
%		\item 
%%		\item 011 数据存储MySQL与MongoDb使用
%%		\item 012 爬虫进阶Selenium的使用
%%		\item 013 爬虫进阶多进程爬虫
%%		\item 014 爬虫进阶多线程爬虫
%%		\item 015 爬虫进阶多协程爬虫
%%		\item 016 爬虫进阶异步爬虫
%%		\item 017 爬虫进阶Scapy爬虫
%%		\item 018 爬虫进阶分布式爬虫
%%		\item 019 爬虫进阶APP爬虫
%%		\item 020 爬虫序章反爬
%	\end{dinglist}
%\end{minipage}
%\end{ascolorbox17}
\frontmatter