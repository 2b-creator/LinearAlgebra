\chapter{特征值理论}
\begin{center}
	% \textcolor[RGB]{255, 0, 0}{\faHeart}所以生命啊,它苦涩如歌.\textcolor[RGB]{255, 0, 0}{\faHeart}
	「人生若只如初见,何事秋风悲画扇」
\end{center}
\rightline{——《木兰花$\cdot$令拟古决绝词》}
\vspace{-5pt}
\begin{center}
	\pgfornament[width=0.36\linewidth,color=lsp]{88}
\end{center}

\section{特征值与特征向量}

\subsection{特征值}

首先我们来定义特征值:

\begin{definition}{特征值(Eigenvalue)}
	若存在向量$v\in V$,线性映射算子$T\in \mathcal{L}(V)$与标量$\lambda \in \mathbb{F}$满足$$T(v)=\lambda v$$则称$\lambda$为$T$的特征值。
\end{definition}

下面是一个例子:

\begin{example}
	若算子$T\in \mathcal{L}(\mathbb{F}^2)$定义为$T((x,y))=(-y,x)$求;

	(1)算子$T$以标准基进行变换的矩阵表示$\mathcal{M}(T)$;

	(2)当$\mathbb{F}=\mathbb{R}$时,验证$T$是否有特征值,如果有则求之,否则说明理由;

	(3)当$\mathbb{F}=\mathbb{C}$时,验证$T$是否有特征值,如果有则求之,否则说明理由。

	\tcblower
	\textcolor{purple}{\textbf{解}}: 
	
	(1) 算子 $ T $ 在标准基下的矩阵表示:

	标准基为 $ e_1 = (1, 0) $ 和 $ e_2 = (0, 1) $。计算 $ T $ 作用在基向量上的结果:
	\begin{itemize}
		\item $ T(e_1) = T((1, 0)) = (0, 1) = e_2 $
		\item $ T(e_2) = T((0, 1)) = (-1, 0) = -e_1 $
	\end{itemize}
	
	因此,矩阵表示为:
	$$
	\mathcal{M}(T) = \begin{pmatrix} 0 & -1 \\ 1 & 0 \end{pmatrix}
	$$
	所以答案是:
	$$
	\begin{pmatrix} 0 & -1 \\ 1 & 0 \end{pmatrix}
	$$

	(2) 假设存在实数$\lambda$和非零向量$(x,y)\in\mathbb{R}^2$,使得:
	$$
	T(x,y) = \lambda(x,y) \implies (-y, x) = (\lambda x, \lambda y)
	$$
	这等价于方程组:
	$$
	\begin{cases}
	-y = \lambda x \\
	x = \lambda y
	\end{cases}
	$$
	\begin{enumerate}
		\item 从第一式得 $y = -\lambda x$,代入第二式:
		$$
		x = \lambda (-\lambda x) = -\lambda^2 x
		$$
		\item 整理得 $x(1 + \lambda^2) = 0$。
		\begin{itemize}
			\item 若$x \neq 0$,则 $1 + \lambda^2 = 0$。但在$\mathbb{R}$中无解。 
			\item  若$x = 0$,则第一式给出$y = 0$,与$v$非零矛盾。 
		\end{itemize}
		所以在实数域$\mathbb{R}$中,$T$没有特征值

		\item 假设存在复数$\lambda$和非零向量$(x,y)\in\mathbb{C}^2$,使得:
		$$
		T(x,y) = \lambda(x,y) \implies (-y, x) = (\lambda x, \lambda y)
		$$
		方程组仍为:
		$$
		\begin{cases}
		-y = \lambda x \\
		x = \lambda y
		\end{cases}
		$$
		\begin{enumerate}
			\item 从第一式得 $y = -\lambda x$,代入第二式:$$x = \lambda (-\lambda x) = -\lambda^2 x$$
			\item 整理得 $x(1 + \lambda^2) = 0$。\begin{itemize}
				\item 若$x \neq 0$,则 $1 + \lambda^2 = 0 \implies \lambda = \pm \mathrm{i}$(复数解)。
				\item 若$x = 0$,则$y = -\lambda \cdot 0 = 0$,与$v$非零矛盾。
			\end{itemize}
			\item 验证$\lambda = \mathrm{i}$和$\lambda = -\mathrm{i}$对应的非零向量:
			\begin{itemize}
				\item 对$\lambda = \mathrm{i}$,由$y = -\mathrm{i} x$,特征向量为$(x, -\mathrm{i} x)$($x \neq 0$)。
				\item 对$\lambda = -\mathrm{i}$,由$y = \mathrm{i} x$,特征向量为$(x, \mathrm{i} x)$($x \neq 0$)。
			\end{itemize}
		\end{enumerate}
		所以在复数域$\mathbb{C}$中,$T$的特征值为$\mathrm{i}$和$-\mathrm{i}$。
	\end{enumerate}
\end{example}

根据上面的定义和例子,我们了解到特征值并不是唯一且一定拥有的,并我们要求$v\neq \boldsymbol{0}$这是因为每个标量$\lambda \in \mathbb{F}$都满足$T(\boldsymbol{0})=\lambda \boldsymbol{0}$。

根据一般的算子矩阵$\mathcal{T}$,下面推导矩阵的特征方程:设$\mathbf{A}$是一个$n\times n$方阵,$\mathbf{A}=\mathcal{M}(T)$,$\lambda$为$\mathbf{A}$的特征值,对应的非零特征向量$v$满足:$$\mathbf{A} v = \lambda v$$将等式改写为:$$\mathbf{A} v - \lambda v = \mathbf{0}$$引入单位矩阵$\mathbf{I}$以保持维度一致:$$(\mathbf{A} - \lambda \mathbf{I}) v = \mathbf{0}$$则非零解$v$存在的条件是系数矩阵的行列式为零:$$\det(\mathbf{A} - \lambda \mathbf{I}) = 0$$上述行列式方程即为矩阵$\mathbf{A}$的特征方程,展开后得到一个关于$\lambda$的多项式。

\begin{corollary}
	关于线性映射算子$T$的矩阵表示$\mathcal{M}(T)$的特征值$\lambda$满足$$\det(\mathcal{M}(T) - \lambda \mathbf{I}) = 0$$
\end{corollary}

\begin{example}
	对于$2\times 2$矩阵 $ \mathbf{A} = \begin{pmatrix} a & b \\ c & d \end{pmatrix} $,求其特征方程,并求$a+b=3$,$ad-bc=2$时,矩阵$\mathbf{A}$的特征值.

	\tcblower
	\textcolor{purple}{\textbf{解}}: 对于$2\times 2$矩阵 $ \mathbf{A} = \begin{pmatrix} a & b \\ c & d \end{pmatrix} $特征方程为$$\det\left( \begin{pmatrix} a-\lambda & b \\ c & d-\lambda \end{pmatrix} \right) = (a-\lambda)(d-\lambda) - bc = 0$$展开后得到:$$\lambda^2 - (a+d)\lambda + (ad - bc) = 0$$带入$a+b=3$,$ad-bc=2$可得方程$$\lambda^2 - 3\lambda + 2 = 0$$因式分解可得$(\lambda-1)(\lambda-2)=0$解得$\lambda=1,2$故矩阵$\mathbf{A}$的特征值为$1$和$2$。
\end{example}

对于等式$T(v)=\lambda v$接下来我们逐步引出特征向量的概念,实际上这里的$v$就是特征向量。

\subsection{特征向量}

\begin{definition}{特征向量(Eigenvector)}
	设线性映射算子$T\in\mathcal{L}(V)$其中$V$为一个线性子空间,$\lambda \in \mathbb{F}$是算子$T$的特征值,若能满足等式$T(v)=\lambda v$且$v\neq 0,v\in V$则称$v$为$T$相对于$\lambda$的特征向量。
\end{definition}

下面我们介绍如何求解特征值所对应的特征向量,实际上它们与解线性方程组的方式一致,接下来请大家带着一个问题去看下面的例题,即:

\begin{ascolorbox1}{思考}
	特征向量是只有一个向量构成集合还是有很多个向量构成的集合?
\end{ascolorbox1}

\begin{example}
	算子$T\in\mathcal{L}(\mathbb{R}^2)$满足$T((x,y,z))=(y+z, x+z, x+y)$求;

	(1)算子$T$以标准基进行变换的矩阵表示$\mathcal{M}(T)$;

	(2)求算子$T$的所有特征值,并求出每个特征值对应的特征向量。
	\tcblower
	\textcolor{purple}{\textbf{解}}: 
	(1)算子$T$在标准基下的矩阵表示:
	
	标准基为$e_1 = (1, 0, 0)$,$e_2 = (0, 1, 0)$,$e_3 = (0, 0, 1)$。计算每个基向量在$T$作用下的结果:
	\begin{itemize}
		\item $T(e_1) = (0, 1, 1)$
		\item $T(e_2) = (1, 0, 1)$
		\item $T(e_3) = (1, 1, 0)$
	\end{itemize}
	因此,矩阵$\mathcal{M}(T)$为:$$\begin{pmatrix} 0 & 1 & 1 \\ 1 & 0 & 1 \\ 1 & 1 & 0 \end{pmatrix}$$

	(2)算子$T$的特征值及对应的特征向量:

	首先求解特征方程$\det(\mathcal{M}(T) - \lambda I) = 0$,其中矩阵$\mathcal{M}(T) - \lambda I$为:$$\begin{pmatrix} -\lambda & 1 & 1 \\ 1 & -\lambda & 1 \\ 1 & 1 & -\lambda \end{pmatrix}$$
	计算行列式:$$\det(\mathcal{M}(T) - \lambda I) = -\lambda^3 + 3\lambda + 2 = 0$$
	解得特征值为$\lambda = 2$和$\lambda = -1$(二重根)。
	\begin{itemize}
		\item 对于$\lambda = 2$,解方程组$(T - 2I)v = 0$,得到特征向量为所有非零标量倍的$(1, 1, 1)$即
		\item 对于$\lambda = -1$,解方程组$(T + I)v = 0$,得到特征向量为满足$x + y + z = 0$的所有非零向量,例如$(1, -1, 0)$和$(1, 0, -1)$。
	\end{itemize}
	计算行列式:$$\det(\mathcal{M}(T) - \lambda I) = -\lambda^3 + 3\lambda + 2 = 0$$解得特征值为$\lambda = 2$和$\lambda = -1$(二重根)。

	\begin{enumerate}
		\item 当特征值 $\lambda = 2$ 时$$\mathcal{M}(T) - 2I = \begin{pmatrix} -2 & 1 & 1 \\ 1 & -2 & 1 \\ 1 & 1 & -2 \end{pmatrix}$$对其进行使用初等变换$$ \begin{pmatrix} -2 & 1 & 1 \\ 1 & -2 & 1 \\ 1 & 1 & -2 \end{pmatrix}\xrightarrow{r_1\leftrightarrow r_2}\begin{pmatrix} 1 & -2 & 1 \\ -2 & 1 & 1 \\ 1 & 1 & -2 \end{pmatrix}\xrightarrow{r_2+2\times r_1}\begin{pmatrix} 1 & -2 & 1 \\ 0 & -3 & 3 \\ 3 & 1 & -2 \end{pmatrix}\xrightarrow{r_3-r_1}\begin{pmatrix} 1 & -2 & 1 \\ 0 & -3 & 3 \\ 0 & 3 & -3 \end{pmatrix}$$$$\xrightarrow{r_3+ r_2}\begin{pmatrix} 1 & -2 & 1 \\ 0 & -3 & 3 \\ 0 & 0 & 0 \end{pmatrix}\xrightarrow{r_2/3}\begin{pmatrix} 1 & -2 & 1 \\ 0 & -1 & 1 \\ 0 & 0 & 0 \end{pmatrix}$$还原为齐次线性方程组$$\left\{\begin{matrix} 
  		x_1-2x_2+x_3=0 \\  
  		-x_2+x_3=0 
		\end{matrix}\right. $$确定基础解系,设$x_1=1$则$x_2=1,x_3=1$由于$r(\mathcal{M}(T))=2$所以$n-r=1$故基础解系中一个$\mathcal{B}$共一个向量即$\mathcal{B}=\left\{ (1,1,1) \right\}$所以当特征值$\lambda=2$时,特征向量为$v=k(1,1,1),k\in \mathbb{R}$
		\item 当特征值 $\lambda = -1$时$$\mathcal{M}(T) + I = \begin{pmatrix} 1 & 1 & 1 \\ 1 & 1 & 1 \\ 1 & 1 & 1 \end{pmatrix}$$所有行相同,化为:$$\begin{pmatrix} 1 & 1 & 1 \\ 0 & 0 & 0 \\ 0 & 0 & 0 \end{pmatrix}$$其对应的方程为$$x+y+z=0$$最后令$x=0,y=1$则$z=-1$令$x=1,y=0$则$z=-1$那么基础解系$\mathcal{B}=\left\{ (0,1,-1),(1,0,-1) \right\}$特征向量为$v=s(-1, 1, 0) + t(-1, 0, 1)$
	\end{enumerate}
\end{example}

接下来回答上述的思考题:特征向量是只有一个向量构成集合还是有很多个向量构成的集合?答案是很多个向量构成的集合,因为最后此类问题求解特征向量都会转化为求齐次线性方程组,然而齐次线性方程组要么全为0解,要么有无数个非零解,全为0解的时,特征值为0或不存在故不存在特征向量。

\section{迹与上三角矩阵}

\subsection{矩阵的迹}

首先我们先定义矩阵的迹。

\begin{definition}{矩阵的迹(trace)}
		$n\times n$矩阵的迹(trace)是矩阵对角元素的和,通常表示为 $\text{Tr}(\mathbf{A})$;若方阵$\mathbf{A}=\begin{pmatrix}  
  a_{11}& a_{12} & \cdots & a_{1n} \\  
	a_{21}& a_{22} & \cdots & a_{2n} \\ 
  \vdots & \vdots & \ddots & \vdots\\  
  a_{n1} & a_{n2} & \cdots &a_{nn}  
\end{pmatrix} $,则$$\text{Tr}(\mathbf{A}):=\sum_{i=1}^{n}a_{ii} $$即为A的所有对角元的和。
\end{definition}

下面是有关迹的一些推论:

\begin{corollary}
	设$\mathbf{A}$是$m\times n$矩阵,$\mathbf{B}$是$n\times m$矩阵。证明:$$\text{Tr}\mathbf{A}\mathbf{B}=\text{Tr}\mathbf{B}\mathbf{A}$$ 
\end{corollary}

\begin{proof}
	考虑$\mathbf{A}\mathbf{B}$的迹,其定义为:
$$
\text{Tr}(\mathbf{A}\mathbf{B}) = \sum_{i=1}^m [\mathbf{A}\mathbf{B}]_{ii} = \sum_{i=1}^m \sum_{k=1}^n a_{ik} b_{ki}
$$

同样地,考虑$\mathbf{B}\mathbf{A}$的迹,其定义为:
$$
\text{Tr}(\mathbf{B}\mathbf{A}) = \sum_{k=1}^n [\mathbf{B}\mathbf{A}]_{kk} = \sum_{k=1}^n \sum_{i=1}^m b_{ki} a_{ik}
$$

接下来,我们交换双重求和的顺序。对于$\text{Tr}(\mathbf{A}\mathbf{B})$,我们可以将求和顺序交换为:
$$
\sum_{i=1}^m \sum_{k=1}^n a_{ik} b_{ki} = \sum_{k=1}^n \sum_{i=1}^m a_{ik} b_{ki}
$$

对于$\text{Tr}(\mathbf{B}\mathbf{A})$,我们注意到$b_{ki} a_{ik} = a_{ik} b_{ki}$,因此:
$$
\sum_{k=1}^n \sum_{i=1}^m b_{ki} a_{ik} = \sum_{k=1}^n \sum_{i=1}^m a_{ik} b_{ki}
$$

显然,$\text{Tr}(\mathbf{A}\mathbf{B})$和$\text{Tr}(\mathbf{B}\mathbf{A})$的表达式相同,只是求和顺序不同,因此它们的值相等。

通过具体例子验证,例如当$\mathbf{A}$是行向量,$\mathbf{B}$是列向量时,或者当$\mathbf{A}$和$\mathbf{B}$分别为不同维度的矩阵时,它们的迹仍然相等。

综上所述,我们证明了:
$$
\text{Tr}(\mathbf{A}\mathbf{B}) = \text{Tr}(\mathbf{B}\mathbf{A})
$$
\begin{flushright}
		$\square$
	\end{flushright}
\end{proof}

\begin{corollary}
	\label{cor:trace}
	设$n$阶方阵$\mathbf{A}$的特征方程为多项式$P(\lambda)=\lambda^n+a_1\lambda^{n-1}+a_2\lambda^{n-2}+\cdots+a_n$,则$$a_1=-\text{Tr}\mathbf{A}$$
\end{corollary}

\begin{proof}
	特征方程多项式可以表示为$P(\lambda) = \det(\lambda \mathbf{I} - \mathbf{A})$,其中$\mathbf{I}$是单位矩阵。设$\mathbf{A}$的特征值为$\lambda_1, \lambda_2, \ldots, \lambda_n$,则特征方程多项式可以分解为:
   $$
   P(\lambda) = (\lambda - \lambda_1)(\lambda - \lambda_2) \cdots (\lambda - \lambda_n)
   $$
   根据韦达定理,$\lambda^{n-1}$项的系数由所有特征值的和决定,即:
   $$
   a_1 = -(\lambda_1 + \lambda_2 + \cdots + \lambda_n)
   $$
   \begin{flushright}
		$\square$
	\end{flushright}
\end{proof}

\subsection{上三角矩阵}

首先我们先定义上三角矩阵。

\begin{definition}{上三角矩阵(upper-triangular matrix)}
	若方阵$\mathbf{A}$主对角线下方的元素全为0则$\mathbf{A}$为上三角矩阵。例如$$\begin{pmatrix}
 \textcolor{blue}{*}  & * & * & *\\
 \textcolor{red}{0}  & \textcolor{blue}{*} & * & *\\
 \textcolor{red}{0} & \textcolor{red}{0} & \textcolor{blue}{*} & *\\
 \textcolor{red}{0} & \textcolor{red}{0} & \textcolor{red}{0} & \textcolor{blue}{*}
\end{pmatrix}$$该矩阵\textcolor{blue}{蓝色}标注的为矩阵对角线。
\end{definition}

在此我们引入上三角矩阵,是为了研究基于算子的某一个基的表示为上三角矩阵可以快速确定一些特征值的性质,以及为了后面的一些定理做一些铺垫。

线性映射算子$T\in\mathcal{L}(V)$的矩阵$\mathcal{M}(T)$依赖于其基的选取,我们设空间$V$的一个基$S=\left\{ v_1,v_2,v_3,\cdots,v_n \right\}$,则$\text{dim}V=n$,不难看出选取基$S$的的矩阵$\mathcal{M}(T)$为一个$n\times n$矩阵,下面我们看一个例子:

若$T\in \mathcal{L}(\mathbb{F}^3)$表示为$T((x,y,z))=(x+y+3z,3y+2z,4z)$则选取$\mathbb{F}^3$的标准基$\left\{ (1,0,0),(0,1,0),(0,0,1) \right\}$的矩阵表示为$$\mathcal{M}(T)=\begin{pmatrix}
 1 & 1 & 3\\
 0 & 3 & 2\\
 0 & 0 & 4
\end{pmatrix}$$事实上,在本章节我们会强调不是以标准基的线性映射矩阵,接下来我们约定一个记号,上述的例子我们用$\mathcal{M}(T)$来默认表示它们是基于标准基的变换矩阵。下面给出基于非标准基的线性映射的矩阵相关记号与定义。
\begin{definition}{基于非标准基的线性映射的矩阵}
	若线性映射算子$T\in\mathcal{V}$,若$S$为线性空间$V$的一个基,那么基于$S$的线性映射$T$的矩阵矩阵表示为$\mathcal{M}(T,S)$,若$S$为线性空间$V$的标准基,则简写为$\mathcal{M}(T)$
\end{definition}

接下来根据上述知识我们可以得到一些推论:

\begin{corollary}
	\label{cor:tri}
	若算子$T\in\mathcal{L}(\mathbb{C}^n)$则一定存在基$S$使得算子关于该基的矩阵$\mathcal{M}(T,S)$为上三角矩阵。
\end{corollary}

\begin{proof}
	为了证明任何线性算子 $ T \in \mathcal{L}(\mathbb{C}^n) $ 存在一个基 $ S $,使得其矩阵表示为上三角矩阵,我们可以使用数学归纳法:当 $ n=1 $ 时,任何 $ 1 \times 1 $ 矩阵本身就是上三角矩阵,结论显然成立。

	那么假设对于所有 $ n-1 $ 维复向量空间上的线性算子,存在一个基使得其矩阵为上三角矩阵。设 $ V $ 为 $ n $ 维复向量空间,$ T \in \mathcal{L}(V) $。由于 $ V $ 是复空间,$ T $ 必有特征值 $ \lambda_1 $,对应非零特征向量 $ v_1 $。令 $ U = \text{Span}(v_1) $,则 $ U $ 是 $ T $-不变的一维子空间。考虑线性映射 $ T - \lambda_1 I $,根据秩零化度定理:$$\dim V = \dim \text{null}(T - \lambda_1 I) + \dim \text{range}(T - \lambda_1 I)$$由于 $ \dim \text{null}(T - \lambda_1 I) \geq 1 $,得 $ \dim \text{range}(T - \lambda_1 I) \leq n - 1 $。选取 $ V $ 的补空间 $ W $ 使得 $ V = U \oplus W $。这里 $ W $ 是 $ n-1 $ 维子空间。定义投影 $ \pi: V \to W $,将任意 $ v \in V $ 分解为 $ v = u + w $(其中 $ u \in U, w \in W $),并令 $ \pi(v) = w $。定义线性算子 $ S: W \to W $ 为 $ S(w) = \pi(T(w)) $。由线性性保证,$ S $ 是 $ W $ 上的线性算子。根据归纳假设,存在 $ W $ 的一组基 $ v_2, \dots, v_n $,使得 $ S $ 在此基下的矩阵为上三角形式。即对任意 $ 2 \leq j \leq n $,有:$$S(v_j) \in \text{Span}(\{v_2, \dots, v_j\}).$$合并基 $ v_1, v_2, \dots, v_n $。对每个 $ v_j $($ j \geq 2 $),有:  
$$
T(v_j) = \pi(T(v_j)) + u_j \quad (u_j \in U).
$$
由于 $ \pi(T(v_j)) = S(v_j) \in \text{Span}(\{v_2, \dots, v_j\}) $,且 $ u_j \in \text{Span}(\{v_1\}) $,因此:  
$$
T(v_j) \in \text{Span}(\{v_1, v_2, \dots, v_j\}).
$$
对于 $ v_1 $,显然 $ T(v_1) = \lambda_1 v_1 \in \text{Span}(\{v_1\}) $。  
因此,基 $ \{v_1, v_2, \dots, v_n\} $ 使得 $ \mathcal{M}(T, S) $ 为上三角矩阵。

由数学归纳法,复向量空间上的任意线性算子均存在一组基,使其矩阵表示为上三角矩阵。  
即,存在基 $ S $ 使得 $ \mathcal{M}(T, S) $ 为上三角矩阵。
\begin{flushright}
		$\square$
	\end{flushright}
\end{proof}

\section{相似变换}

\subsection{相似矩阵}

首先给出矩阵相似的定义。

\begin{definition}{相似矩阵}
	若$n\times n$矩阵$\mathbf{A}$与矩阵$\mathbf{B}$相似,则存在 $n\times n$ 可逆矩阵$\mathbf{P}$使得$$\mathbf{A}=\mathbf{P}\mathbf{B}\mathbf{P}^{-1}$$记作$\mathbf{A}\sim \mathbf{B}$
\end{definition}

根据上述定义,我们不难发现如下的推论:

\begin{corollary}
	相似矩阵的性质:
	\begin{enumerate}
		\item 自反性:$\mathbf{A}$与$\mathbf{A}$相似;
		\item 对称性:若$\mathbf{A}$与$\mathbf{B}$相似,则$\mathbf{B}$与$\mathbf{A}$相似;
		\item 传递性:若$\mathbf{A}$与$\mathbf{B}$相似,$\mathbf{B}$与$\mathbf{C}$相似,则$\mathbf{A}$与$\mathbf{C}$相似
	\end{enumerate}
\end{corollary}

\begin{proof}
	证明留给读者,参见课后练习 B 组。
	\begin{flushright}
		$\square$
	\end{flushright}
\end{proof}

\begin{corollary}
	如果矩阵$\mathbf{A}\sim \mathbf{B}$,则它们的特征方程$p(\lambda)$相同,并满足$\text{Tr}\mathbf{A}=\text{Tr}\mathbf{B}$与$\det \mathbf{A}=\det \mathbf{B}$
\end{corollary}

\begin{proof}
	假设矩阵$\mathbf{A}$和$\mathbf{B}$相似,则存在可逆矩阵$\mathbf{P}$使得$\mathbf{B} = \mathbf{P}^{-1}\mathbf{A}\mathbf{P}$。

首先,证明它们的特征方程相同:

计算矩阵$\mathbf{B}$的特征方程多项式:
$$
\det(\lambda \mathbf{I} - \mathbf{B}) = \det(\lambda \mathbf{I} - \mathbf{P}^{-1}\mathbf{A}\mathbf{P})
$$
将$\lambda \mathbf{I}$改写为$\mathbf{P}^{-1}(\lambda \mathbf{I})\mathbf{P}$,因为$\mathbf{P}^{-1}(\lambda \mathbf{I})\mathbf{P} = \lambda \mathbf{I}$。于是:
$$
\det(\lambda \mathbf{I} - \mathbf{P}^{-1}\mathbf{A}\mathbf{P}) = \det(\mathbf{P}^{-1}(\lambda \mathbf{I} - \mathbf{A})\mathbf{P})
$$
利用行列式的乘法性质:
$$
\det(\mathbf{P}^{-1}(\lambda \mathbf{I} - \mathbf{A})\mathbf{P}) = \det(\mathbf{P}^{-1})\det(\lambda \mathbf{I} - \mathbf{A})\det(\mathbf{P})
$$
由于$\det(\mathbf{P}^{-1})\det(\mathbf{P}) = \det(\mathbf{P}^{-1}\mathbf{P}) = \det(\mathbf{I}) = 1$,因此:
$$
\det(\lambda \mathbf{I} - \mathbf{B}) = \det(\lambda \mathbf{I} - \mathbf{A})
$$
即$\mathbf{A}$和$\mathbf{B}$的特征方程相同。

接下来,由于特征方程相同,它们的特征方程多项式相同。特征方程多项式的形式为:
$$
\lambda^n - (\text{Tr}\mathbf{A})\lambda^{n-1} + \cdots + (-1)^n \det{\mathbf{A}}\footnote{为什么最后一项是$(-1)^n \det{\mathbf{A}}$?}
$$
比较多项式两边的对应系数:

第二项系数\footnote{根据推论\ref{cor:trace}}:$-\text{Tr}\mathbf{A} = -\text{Tr}\mathbf{B}$,因此$\text{Tr}\mathbf{A} = \text{Tr}\mathbf{B}$。常数项:$(-1)^n \det{\mathbf{A}} = (-1)^n \det{\mathbf{B}}$,因此$\det{\mathbf{A}} = \det{\mathbf{B}}$。

因此,矩阵$\mathbf{A}$和$\mathbf{B}$的迹和行列式相等。
\begin{flushright}
		$\square$
	\end{flushright}
\end{proof}

\begin{corollary}
	\label{cor:tri2}
	任意的$n$阶方阵$\mathbf{A}$都能相似于一个上三角矩阵。
\end{corollary}

\begin{proof}	
	设 $\mathbf{A}$ 是任意 $n \times n$ 复矩阵。$\mathbf{A}$ 可视为线性算子 $T_A: \mathbb{C}^n \to \mathbb{C}^n$ 在标准基 $B$ 下的矩阵表示,即 $\mathcal{M}(T_A, B) = \mathbf{A}$。  
	由推论\ref{cor:tri}证明,存在基 $S$ 使得 $\mathcal{M}(T_A, S)$ 是上三角矩阵。基变换矩阵 $\mathbf{P}$ 满足 $\mathbf{P}$ 的列是 $S$ 在基 $B$ 下的坐标向量,且  
	$$
	\mathcal{M}(T_A, S) = \mathbf{P}^{-1} \mathbf{A} \mathbf{P}.
	$$
	因此,$\mathbf{P}^{-1} \mathbf{A} \mathbf{P}$ 是上三角矩阵,即 $\mathbf{A}$ 相似于一个上三角矩阵。\begin{flushright}
		$\square$
	\end{flushright}
\end{proof}

\subsection{Hamilton-Cayley 定理}

根据上述的铺垫我们引入并证明 Hamilton-Cayley 定理

\begin{theorem}{Hamilton-Cayley 定理}
	多项式$P(\lambda)=\lambda^n+a_1\lambda^{n-1}+a_2\lambda^{n-2}+\cdots+a_n$为$n$阶方阵$\mathbf{A}$的特征方程,那么矩阵运算$$P(\mathbf{A}):=\mathbf{A}^n+a_1\mathbf{A}^{n-1}+a_2\mathbf{A}^{n-2}+\cdots+a_n\mathbf{I}=\mathbf{O}$$成立,$P(\mathbf{A})$为零矩阵$\mathbf{O}$。
\end{theorem}

这个定理挺奇妙,我们下面尝试一步步证明它。

\begin{proof}
	特征方程多项式将其可以改写为
	$$
   	P(\lambda) = (\lambda - \lambda_1)(\lambda - \lambda_2) \cdots (\lambda - \lambda_n)
   	$$根据推论\ref{cor:tri2},存在可逆矩阵$\mathbf{P}$,使得$$\mathbf{P}^{-1} \mathbf{A} \mathbf{P}=\begin{pmatrix}
 \lambda_1 & * & * & \cdots & *\\
  & \lambda_2 & * & \cdots & *\\
  &  & \lambda_3 & \cdots & *\\
  &  &  & \ddots & \vdots \\ 
  &  &  &  & \lambda_n
\end{pmatrix}$$将$\mathbf{P}^{-1} \mathbf{A} \mathbf{P}$代入矩阵运算即$$P(\mathbf{P}^{-1} \mathbf{A} \mathbf{P})=(\mathbf{P}^{-1} \mathbf{A} \mathbf{P} - \lambda_1\mathbf{I})(\mathbf{P}^{-1} \mathbf{A} \mathbf{P} - \lambda_2\mathbf{I}) \cdots (\mathbf{P}^{-1} \mathbf{A} \mathbf{P} - \lambda_n\mathbf{I})$$我们将其写成矩阵乘积的形式即\begin{tiny}
$$\begin{pmatrix}
 0 & * & * & \cdots & *\\
  & \lambda_2-\lambda_1 & * & \cdots & *\\
  &  & \lambda_3-\lambda_1 & \cdots & *\\
  &  &  & \ddots & \vdots \\ 
  &  &  &  & \lambda_n-\lambda_1
\end{pmatrix}\begin{pmatrix}
  \lambda_1-\lambda_2 & * & * & \cdots & *\\
  & 0 & * & \cdots & *\\
  &  & \lambda_3-\lambda_2 & \cdots & *\\
  &  &  & \ddots & \vdots \\ 
  &  &  &  & \lambda_n-\lambda_2
\end{pmatrix}\cdots\begin{pmatrix}
  \lambda_1-\lambda_n & * & * & \cdots & *\\
  & \lambda_2-\lambda_n & * & \cdots & *\\
  &  & \lambda_3-\lambda_n & \cdots & *\\
  &  &  & \ddots & \vdots \\ 
  &  &  &  & 0
\end{pmatrix}=\mathbf{O}$$
\end{tiny}由于单位矩阵$\mathbf{I}=\mathbf{P}\mathbf{P}^{-1}$可得$$P(\mathbf{P}^{-1} \mathbf{A} \mathbf{P})=(\mathbf{P}^{-1} \mathbf{A} \mathbf{P} - \lambda_1\mathbf{P}\mathbf{P}^{-1})(\mathbf{P}^{-1} \mathbf{A} \mathbf{P} - \lambda_2\mathbf{P}\mathbf{P}^{-1}) \cdots (\mathbf{P}^{-1} \mathbf{A} \mathbf{P} - \lambda_n\mathbf{P}\mathbf{P}^{-1})$$提出$\mathbf{P}\mathbf{P}^{-1}$部分可得$\mathbf{P}P(\mathbf{A})\mathbf{P}^{-1}=\mathbf{O}$即$P(\mathbf{A})=\mathbf{O}$
\begin{flushright}
		$\square$
	\end{flushright}
\end{proof}

Hamilton-Cayley 定理在很多方面也有应用,例如求矩阵多项式,求逆矩阵,伴随矩阵;下面我们就求逆矩阵讲讲如何利用 Hamilton-Cayley 定理求解。

设$n\times n$矩阵$\mathbf{A}$可逆,首先我们根据$n\times n$可逆矩阵的秩等于$n$,可得其特征方程多项式为$$P(\lambda)=\lambda^n+a_1\lambda^{n-1}+a_2\lambda^{n-2}+\cdots+a_n$$且$a_n=(-1)^n \det \mathbf{A}\neq 0$由Hamilton-Cayley定理,我们有等式$$\mathbf{A}^n+a_1\mathbf{A}^{n-1}+a_2\mathbf{A}^{n-2}+\cdots+a_n\mathbf{I}=\mathbf{O}$$化简这个式子我们可以得到$$-\frac{1}{a_n}\left( \mathbf{A}^n+a_1\mathbf{A}^{n-1}+a_2\mathbf{A}^{n-2}+\cdots a_{n-1}\mathbf{A} \right)=\mathbf{I}$$由此可得$$\mathbf{A}^{-1}=-\frac{1}{a_n}\left( \mathbf{A}^{n-1}+a_1\mathbf{A}^{n-2}+a_2\mathbf{A}^{n-3}+\cdots a_{n-1}\right)$$

\begin{example}
	使用 Hamilton-Cayley 定理计算方阵$\mathbf{A}=\begin{pmatrix}
 1 & 2 & 4\\
 0 & 1 & -1\\
 2 & 0 & 1
\end{pmatrix}$的逆$\mathbf{A}^{-1}$
\tcblower
\textcolor{purple}{\textbf{解}}: $\mathbf{A}$的特征方程多项式为$$P(\lambda)=\det \left( \lambda\mathbf{I}-\mathbf{A} \right)=-\lambda^3+3\lambda^2+5\lambda-11$$所以可得其逆矩阵$$\mathbf{A}^{-1}=\frac{1}{11}\left( -\mathbf{A}^2+3\mathbf{A}+5\mathbf{I} \right)=\frac{1}{11}\begin{pmatrix}
 1 & 2 & 6\\
 2 & 7 & -1\\
 2 & -4 & -1
\end{pmatrix}$$
\end{example}

\subsection{对角化矩阵}

首先给出对角矩阵的定义:

\begin{definition}{对角矩阵(Diagonal Matrix)与矩阵的对角化}
	若$n\times n$方阵$\mathbf{A}$的元素$a_{ij}$满足$a_{ij}=c_i\delta_{ij}$,其中$c_i\in \mathbb{F}$,$\delta_{ij}$为函数表示为$$\delta_{ij}=\left\{\begin{matrix} 
  1 \quad i=j \\  
  0 \quad i\neq j
\end{matrix}\right. \footnote{Kronecker Delta, Kronecker function, 克罗内克函数, https://mathworld.wolfram.com/KroneckerDelta.html}$$则该矩阵为对角矩阵,即该矩阵的元素除了其对角线上的元素以外均为 0,并记作$\text{diag}(a_{1},a_2,\cdots,a_n)$,即$$\text{diag}(a_{1},a_2,\cdots,a_n):=\begin{pmatrix}
 a_1 & 0 & \cdots &0 \\
 0 & a_2 & \cdots & 0 \\
 \vdots & \vdots & \ddots & \vdots\\
 0 & 0 & \cdots & a_1
\end{pmatrix}$$如果方阵$\mathbf{B}$与一个对角矩阵相似,则称$\mathbf{B}$称为可对角化。
\end{definition}

若方阵$\mathbf{A}\sim \text{diag}(a_{1},a_2,\cdots,a_n)$,则$\mathbf{A}$的特征方程$p(\lambda)=(\lambda-a_1)(\lambda-a_2)\cdots(\lambda-a_n)$因此这个对角矩阵的元素即为$\mathbf{A}$的特征值。
下面我们来描述矩阵可对角化的充分必要条件,若 $\mathbf{A}$ 可对角化,那么则有式子成立$$\mathbf{A}=\mathbf{P}\text{diag}(a_{1},a_2,\cdots,a_n)\mathbf{P}^{-1}$$设$\mathbf{P}$矩阵由列向量$\left\{ b_1,b_2,b_3,\cdots,b_n \right\}$构成,则$\mathbf{P}=\begin{pmatrix}
	b_1 & b_2 & b_3 & \cdots & b_n
\end{pmatrix}$由于$\mathbf{P}$可逆,所以其秩$r(\mathbf{P})=n$,因此$\left\{ b_1,b_2,b_3,\cdots,b_n \right\}$为其极大线性无关组,那么就可以写作下面的形式$$\mathbf{A}\begin{pmatrix}
	b_1 & b_2 & b_3 & \cdots & b_n
\end{pmatrix}=\begin{pmatrix}
	b_1 & b_2 & b_3 & \cdots & b_n
\end{pmatrix}\text{diag}(a_{1},a_2,\cdots,a_n)$$设矩阵 $\mathbf{B} = \begin{pmatrix} b_1 & b_2 & \cdots & b_n \end{pmatrix}$,其列由向量 $b_1, b_2, \ldots, b_n$ 组成。对角矩阵 $\mathbf{D} = \text{diag}(a_1, a_2, \cdots, a_n)$。则等式可写为:
$$
\mathbf{A} \mathbf{B} = \mathbf{B} \mathbf{D}
$$
\begin{itemize}
	\item 左边 $\mathbf{A} \mathbf{B}$ 的结果是一个矩阵,其第 $j$ 列为 $\mathbf{A} b_j$,即:
  $$
  \mathbf{A} \mathbf{B} = \begin{pmatrix} \mathbf{A} b_1 & \mathbf{A} b_2 & \cdots & \mathbf{A} b_n \end{pmatrix}
  $$
  	\item 右边 $\mathbf{B} \mathbf{D}$ 的结果是一个矩阵,其第 $j$ 列为 $\mathbf{B}$ 的第 $j$ 列乘以 $\mathbf{D}$ 的第 $j$ 个对角元素 $a_j$,即:
  $$
  \mathbf{B} \mathbf{D} = \begin{pmatrix} a_1 b_1 & a_2 b_2 & \cdots & a_n b_n \end{pmatrix}
  $$
\end{itemize}

因此,等式等价于:
$$
\begin{pmatrix} \mathbf{A} b_1 & \mathbf{A} b_2 & \cdots & \mathbf{A} b_n \end{pmatrix} = \begin{pmatrix} a_1 b_1 & a_2 b_2 & \cdots & a_n b_n \end{pmatrix}
$$
这表示对应列相等,即对于每个 $j = 1, 2, \ldots, n$:
$$
\mathbf{A} b_j = a_j b_j
$$

该等式化简后表示:每个列向量 $b_i$ 是矩阵 $\mathbf{A}$ 的特征向量,对应的特征值为 $a_i$。即:
$$
\mathbf{A} b_i = a_i b_i, \quad \forall i = 1, 2, \ldots, n
$$