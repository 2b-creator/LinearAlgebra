\chapter{特征值理论}
\begin{center}
	% \textcolor[RGB]{255, 0, 0}{\faHeart}所以生命啊,它苦涩如歌.\textcolor[RGB]{255, 0, 0}{\faHeart}
	「人生若只如初见,何事秋风悲画扇」
\end{center}
\rightline{——《木兰花$\cdot$令拟古决绝词》}
\vspace{-5pt}
\begin{center}
	\pgfornament[width=0.36\linewidth,color=lsp]{88}
\end{center}

\section{特征值与特征向量}

\subsection{特征值}

首先我们来定义特征值:

\begin{definition}{特征值(Eigenvalue)}
	若存在向量$v\in V$,线性映射算子$T\in \mathcal{L}(V)$与标量$\lambda \in \mathbb{F}$满足$$T(v)=\lambda v$$则称$\lambda$为$T$的特征值。
\end{definition}

下面是一个例子:

\begin{example}
	若算子$T\in \mathcal{L}(\mathbb{F}^2)$定义为$T((x,y))=(-y,x)$求;

	(1)算子$T$以标准基进行变换的矩阵表示$\mathcal{M}(T)$;

	(2)当$\mathbb{F}=\mathbb{R}$时,验证$T$是否有特征值,如果有则求之,否则说明理由;

	(3)当$\mathbb{F}=\mathbb{C}$时,验证$T$是否有特征值,如果有则求之,否则说明理由。

	\tcblower
	\textcolor{purple}{\textbf{解}}: 
	
	(1) 算子 $ T $ 在标准基下的矩阵表示:

	标准基为 $ e_1 = (1, 0) $ 和 $ e_2 = (0, 1) $。计算 $ T $ 作用在基向量上的结果:
	\begin{itemize}
		\item $ T(e_1) = T((1, 0)) = (0, 1) = e_2 $
		\item $ T(e_2) = T((0, 1)) = (-1, 0) = -e_1 $
	\end{itemize}
	
	因此,矩阵表示为:
	$$
	\mathcal{M}(T) = \begin{pmatrix} 0 & -1 \\ 1 & 0 \end{pmatrix}
	$$
	所以答案是:
	$$
	\begin{pmatrix} 0 & -1 \\ 1 & 0 \end{pmatrix}
	$$

	(2) 假设存在实数$\lambda$和非零向量$(x,y)\in\mathbb{R}^2$,使得:
	$$
	T(x,y) = \lambda(x,y) \implies (-y, x) = (\lambda x, \lambda y)
	$$
	这等价于方程组:
	$$
	\begin{cases}
	-y = \lambda x \\
	x = \lambda y
	\end{cases}
	$$
	\begin{enumerate}
		\item 从第一式得 $y = -\lambda x$,代入第二式:
		$$
		x = \lambda (-\lambda x) = -\lambda^2 x
		$$
		\item 整理得 $x(1 + \lambda^2) = 0$。
		\begin{itemize}
			\item 若$x \neq 0$,则 $1 + \lambda^2 = 0$。但在$\mathbb{R}$中无解。 
			\item  若$x = 0$,则第一式给出$y = 0$,与$v$非零矛盾。 
		\end{itemize}
		所以在实数域$\mathbb{R}$中,$T$没有特征值

		\item 假设存在复数$\lambda$和非零向量$(x,y)\in\mathbb{C}^2$,使得:
		$$
		T(x,y) = \lambda(x,y) \implies (-y, x) = (\lambda x, \lambda y)
		$$
		方程组仍为:
		$$
		\begin{cases}
		-y = \lambda x \\
		x = \lambda y
		\end{cases}
		$$
		\begin{enumerate}
			\item 从第一式得 $y = -\lambda x$,代入第二式:$$x = \lambda (-\lambda x) = -\lambda^2 x$$
			\item 整理得 $x(1 + \lambda^2) = 0$。\begin{itemize}
				\item 若$x \neq 0$,则 $1 + \lambda^2 = 0 \implies \lambda = \pm \mathrm{i}$(复数解)。
				\item 若$x = 0$,则$y = -\lambda \cdot 0 = 0$,与$v$非零矛盾。
			\end{itemize}
			\item 验证$\lambda = \mathrm{i}$和$\lambda = -\mathrm{i}$对应的非零向量:
			\begin{itemize}
				\item 对$\lambda = \mathrm{i}$,由$y = -\mathrm{i} x$,特征向量为$(x, -\mathrm{i} x)$($x \neq 0$)。
				\item 对$\lambda = -\mathrm{i}$,由$y = \mathrm{i} x$,特征向量为$(x, \mathrm{i} x)$($x \neq 0$)。
			\end{itemize}
		\end{enumerate}
		所以在复数域$\mathbb{C}$中,$T$的特征值为$\mathrm{i}$和$-\mathrm{i}$。
	\end{enumerate}
\end{example}

根据上面的定义和例子,我们了解到特征值并不是唯一且一定拥有的,并我们要求$v\neq \boldsymbol{0}$这是因为每个标量$\lambda \in \mathbb{F}$都满足$T(\boldsymbol{0})=\lambda \boldsymbol{0}$。

根据一般的算子矩阵$\mathcal{T}$,下面推导矩阵的特征方程:设$\mathbf{A}$是一个$n\times n$方阵,$\mathbf{A}=\mathcal{M}(T)$,$\lambda$为$\mathbf{A}$的特征值,对应的非零特征向量$v$满足:$$\mathbf{A} v = \lambda v$$将等式改写为:$$\mathbf{A} v - \lambda v = \mathbf{0}$$引入单位矩阵$\mathbf{I}$以保持维度一致:$$(\mathbf{A} - \lambda \mathbf{I}) v = \mathbf{0}$$则非零解$v$存在的条件是系数矩阵的行列式为零:$$\det(\mathbf{A} - \lambda \mathbf{I}) = 0$$上述行列式方程即为矩阵$\mathbf{A}$的特征方程,展开后得到一个关于$\lambda$的多项式。

\begin{corollary}
	关于线性映射算子$T$的矩阵表示$\mathcal{M}(T)$的特征值$\lambda$满足$$\det(\mathcal{M}(T) - \lambda \mathbf{I}) = 0$$
\end{corollary}

\begin{example}
	对于$2\times 2$矩阵 $ \mathbf{A} = \begin{pmatrix} a & b \\ c & d \end{pmatrix} $,求其特征方程,并求$a+b=3$,$ad-bc=2$时,矩阵$\mathbf{A}$的特征值.

	\tcblower
	\textcolor{purple}{\textbf{解}}: 对于$2\times 2$矩阵 $ \mathbf{A} = \begin{pmatrix} a & b \\ c & d \end{pmatrix} $特征方程为$$\det\left( \begin{pmatrix} a-\lambda & b \\ c & d-\lambda \end{pmatrix} \right) = (a-\lambda)(d-\lambda) - bc = 0$$展开后得到:$$\lambda^2 - (a+d)\lambda + (ad - bc) = 0$$带入$a+b=3$,$ad-bc=2$可得方程$$\lambda^2 - 3\lambda + 2 = 0$$因式分解可得$(\lambda-1)(\lambda-2)=0$解得$\lambda=1,2$故矩阵$\mathbf{A}$的特征值为$1$和$2$。
\end{example}

对于等式$T(v)=\lambda v$接下来我们逐步引出特征向量的概念,实际上这里的$v$就是特征向量。

\subsection{特征向量}

\begin{definition}{特征向量(Eigenvector)}
	设线性映射算子$T\in\mathcal{L}(V)$其中$V$为一个线性子空间,$\lambda \in \mathbb{F}$是算子$T$的特征值,若能满足等式$T(v)=\lambda v$且$v\neq 0,v\in V$则称$v$为$T$相对于$\lambda$的特征向量。
\end{definition}

下面我们介绍如何求解特征值所对应的特征向量,实际上它们与解线性方程组的方式一致,接下来请大家带着一个问题去看下面的例题,即:

\begin{ascolorbox1}{思考}
	特征向量是只有一个向量构成集合还是有很多个向量构成的集合?
\end{ascolorbox1}

\begin{example}
	算子$T\in\mathcal{L}(\mathbb{R}^2)$满足$T((x,y,z))=(y+z, x+z, x+y)$求;

	(1)算子$T$以标准基进行变换的矩阵表示$\mathcal{M}(T)$;

	(2)求算子$T$的所有特征值,并求出每个特征值对应的特征向量。
	\tcblower
	\textcolor{purple}{\textbf{解}}: 
	(1)算子$T$在标准基下的矩阵表示:
	
	标准基为$e_1 = (1, 0, 0)$,$e_2 = (0, 1, 0)$,$e_3 = (0, 0, 1)$。计算每个基向量在$T$作用下的结果:
	\begin{itemize}
		\item $T(e_1) = (0, 1, 1)$
		\item $T(e_2) = (1, 0, 1)$
		\item $T(e_3) = (1, 1, 0)$
	\end{itemize}
	因此,矩阵$\mathcal{M}(T)$为:$$\begin{pmatrix} 0 & 1 & 1 \\ 1 & 0 & 1 \\ 1 & 1 & 0 \end{pmatrix}$$

	(2)算子$T$的特征值及对应的特征向量:

	首先求解特征方程$\det(\mathcal{M}(T) - \lambda I) = 0$,其中矩阵$\mathcal{M}(T) - \lambda I$为:$$\begin{pmatrix} -\lambda & 1 & 1 \\ 1 & -\lambda & 1 \\ 1 & 1 & -\lambda \end{pmatrix}$$
	计算行列式:$$\det(\mathcal{M}(T) - \lambda I) = -\lambda^3 + 3\lambda + 2 = 0$$
	解得特征值为$\lambda = 2$和$\lambda = -1$(二重根)。
	\begin{itemize}
		\item 对于$\lambda = 2$,解方程组$(T - 2I)v = 0$,得到特征向量为所有非零标量倍的$(1, 1, 1)$即
		\item 对于$\lambda = -1$,解方程组$(T + I)v = 0$,得到特征向量为满足$x + y + z = 0$的所有非零向量,例如$(1, -1, 0)$和$(1, 0, -1)$。
	\end{itemize}
	计算行列式:$$\det(\mathcal{M}(T) - \lambda I) = -\lambda^3 + 3\lambda + 2 = 0$$解得特征值为$\lambda = 2$和$\lambda = -1$(二重根)。

	\begin{enumerate}
		\item 当特征值 $\lambda = 2$ 时$$\mathcal{M}(T) - 2I = \begin{pmatrix} -2 & 1 & 1 \\ 1 & -2 & 1 \\ 1 & 1 & -2 \end{pmatrix}$$对其进行使用初等变换$$ \begin{pmatrix} -2 & 1 & 1 \\ 1 & -2 & 1 \\ 1 & 1 & -2 \end{pmatrix}\xrightarrow{r_1\leftrightarrow r_2}\begin{pmatrix} 1 & -2 & 1 \\ -2 & 1 & 1 \\ 1 & 1 & -2 \end{pmatrix}\xrightarrow{r_2+2\times r_1}\begin{pmatrix} 1 & -2 & 1 \\ 0 & -3 & 3 \\ 3 & 1 & -2 \end{pmatrix}\xrightarrow{r_3-r_1}\begin{pmatrix} 1 & -2 & 1 \\ 0 & -3 & 3 \\ 0 & 3 & -3 \end{pmatrix}$$$$\xrightarrow{r_3+ r_2}\begin{pmatrix} 1 & -2 & 1 \\ 0 & -3 & 3 \\ 0 & 0 & 0 \end{pmatrix}\xrightarrow{r_2/3}\begin{pmatrix} 1 & -2 & 1 \\ 0 & -1 & 1 \\ 0 & 0 & 0 \end{pmatrix}$$还原为齐次线性方程组$$\left\{\begin{matrix} 
  		x_1-2x_2+x_3=0 \\  
  		-x_2+x_3=0 
		\end{matrix}\right. $$确定基础解系,设$x_1=1$则$x_2=1,x_3=1$由于$r(\mathcal{M}(T))=2$所以$n-r=1$故基础解系中一个$\mathcal{B}$共一个向量即$\mathcal{B}=\left\{ (1,1,1) \right\}$所以当特征值$\lambda=2$时,特征向量为$v=k(1,1,1),k\in \mathbb{R}$
		\item 当特征值 $\lambda = -1$时$$\mathcal{M}(T) + I = \begin{pmatrix} 1 & 1 & 1 \\ 1 & 1 & 1 \\ 1 & 1 & 1 \end{pmatrix}$$所有行相同,化为:$$\begin{pmatrix} 1 & 1 & 1 \\ 0 & 0 & 0 \\ 0 & 0 & 0 \end{pmatrix}$$其对应的方程为$$x+y+z=0$$最后令$x=0,y=1$则$z=-1$令$x=1,y=0$则$z=-1$那么基础解系$\mathcal{B}=\left\{ (0,1,-1),(1,0,-1) \right\}$特征向量为$v=s(-1, 1, 0) + t(-1, 0, 1)$
	\end{enumerate}
\end{example}

接下来回答上述的思考题:特征向量是只有一个向量构成集合还是有很多个向量构成的集合?答案是很多个向量构成的集合,因为最后此类问题求解特征向量都会转化为求齐次线性方程组,然而齐次线性方程组要么全为0解,要么有无数个非零解,全为0解的时,特征值为0或不存在故不存在特征向量。

\section{迹与上三角矩阵}

\subsection{矩阵的迹}

首先我们先定义矩阵的迹。

\begin{definition}{矩阵的迹(trace)}
		$n\times n$矩阵的迹(trace)是矩阵对角元素的和,通常表示为 $\text{Tr}(\mathbf{A})$;若方阵$\mathbf{A}=\begin{pmatrix}  
  a_{11}& a_{12} & \cdots & a_{1n} \\  
	a_{21}& a_{22} & \cdots & a_{2n} \\ 
  \vdots & \vdots & \ddots & \vdots\\  
  a_{n1} & a_{n2} & \cdots &a_{nn}  
\end{pmatrix} $,则$$\text{Tr}(\mathbf{A}):=\sum_{i=1}^{n}a_{ii} $$即为A的所有对角元的和。
\end{definition}

下面是有关迹的一些推论:

\begin{corollary}
	设$\mathbf{A}$是$m\times n$矩阵,$\mathbf{B}$是$n\times m$矩阵。证明:$$\text{Tr}\mathbf{A}\mathbf{B}=\text{Tr}\mathbf{B}\mathbf{A}$$ 
\end{corollary}

\begin{proof}
	考虑$\mathbf{A}\mathbf{B}$的迹,其定义为:
$$
\text{Tr}(\mathbf{A}\mathbf{B}) = \sum_{i=1}^m (\mathbf{A}\mathbf{B})_{ii} = \sum_{i=1}^m \sum_{k=1}^n a_{ik} b_{ki}
$$

同样地,考虑$\mathbf{B}\mathbf{A}$的迹,其定义为:
$$
\text{Tr}(\mathbf{B}\mathbf{A}) = \sum_{k=1}^n (\mathbf{B}\mathbf{A})_{kk} = \sum_{k=1}^n \sum_{i=1}^m b_{ki} a_{ik}
$$

接下来,我们交换双重求和的顺序。对于$\text{Tr}(\mathbf{A}\mathbf{B})$,我们可以将求和顺序交换为:
$$
\sum_{i=1}^m \sum_{k=1}^n a_{ik} b_{ki} = \sum_{k=1}^n \sum_{i=1}^m a_{ik} b_{ki}
$$

对于$\text{Tr}(\mathbf{B}\mathbf{A})$,我们注意到$b_{ki} a_{ik} = a_{ik} b_{ki}$,因此:
$$
\sum_{k=1}^n \sum_{i=1}^m b_{ki} a_{ik} = \sum_{k=1}^n \sum_{i=1}^m a_{ik} b_{ki}
$$

显然,$\text{Tr}(\mathbf{A}\mathbf{B})$和$\text{Tr}(\mathbf{B}\mathbf{A})$的表达式相同,只是求和顺序不同,因此它们的值相等。

通过具体例子验证,例如当$\mathbf{A}$是行向量,$\mathbf{B}$是列向量时,或者当$\mathbf{A}$和$\mathbf{B}$分别为不同维度的矩阵时,它们的迹仍然相等。

综上所述,我们证明了:
$$
\text{Tr}(\mathbf{A}\mathbf{B}) = \text{Tr}(\mathbf{B}\mathbf{A})
$$
\begin{flushright}
		$\square$
	\end{flushright}
\end{proof}

\begin{corollary}
	\label{cor:trace}
	设$n$阶方阵$\mathbf{A}$的特征方程为多项式$P(\lambda)=\lambda^n+a_1\lambda^{n-1}+a_2\lambda^{n-2}+\cdots+a_n$,则$$a_1=-\text{Tr}\mathbf{A}$$
\end{corollary}

\begin{proof}
	特征方程多项式可以表示为$P(\lambda) = \det(\lambda \mathbf{I} - \mathbf{A})$,其中$\mathbf{I}$是单位矩阵。设$\mathbf{A}$的特征值为$\lambda_1, \lambda_2, \ldots, \lambda_n$,则特征方程多项式可以分解为:
   cc
   根据韦达定理,$\lambda^{n-1}$项的系数由所有特征值的和决定,即:
   $$
   a_1 = -(\lambda_1 + \lambda_2 + \cdots + \lambda_n)
   $$
   \begin{flushright}
		$\square$
	\end{flushright}
\end{proof}

\subsection{上三角矩阵}

首先我们先定义上三角矩阵。

\begin{definition}{上三角矩阵(upper-triangular matrix)}
	若方阵$\mathbf{A}$主对角线下方的元素全为0则$\mathbf{A}$为上三角矩阵。例如$$\begin{pmatrix}
 \textcolor{blue}{*}  & * & * & *\\
 \textcolor{red}{0}  & \textcolor{blue}{*} & * & *\\
 \textcolor{red}{0} & \textcolor{red}{0} & \textcolor{blue}{*} & *\\
 \textcolor{red}{0} & \textcolor{red}{0} & \textcolor{red}{0} & \textcolor{blue}{*}
\end{pmatrix}$$该矩阵\textcolor{blue}{蓝色}标注的为矩阵对角线。
\end{definition}

在此我们引入上三角矩阵,是为了研究基于算子的某一个基的表示为上三角矩阵可以快速确定一些特征值的性质,以及为了后面的一些定理做一些铺垫。

线性映射算子$T\in\mathcal{L}(V)$的矩阵$\mathcal{M}(T)$依赖于其基的选取,我们设空间$V$的一个基$S=\left\{ v_1,v_2,v_3,\cdots,v_n \right\}$,则$\text{dim}V=n$,不难看出选取基$S$的的矩阵$\mathcal{M}(T)$为一个$n\times n$矩阵,下面我们看一个例子:

若$T\in \mathcal{L}(\mathbb{F}^3)$表示为$T((x,y,z))=(x+y+3z,3y+2z,4z)$则选取$\mathbb{F}^3$的标准基$\left\{ (1,0,0),(0,1,0),(0,0,1) \right\}$的矩阵表示为$$\mathcal{M}(T)=\begin{pmatrix}
 1 & 1 & 3\\
 0 & 3 & 2\\
 0 & 0 & 4
\end{pmatrix}$$事实上,在本章节我们会强调不是以标准基的线性映射矩阵,接下来我们约定一个记号,上述的例子我们用$\mathcal{M}(T)$来默认表示它们是基于标准基的变换矩阵。下面给出基于非标准基的线性映射的矩阵相关记号与定义。
\begin{definition}{基于非标准基的线性映射的矩阵}
	若线性映射算子$T\in\mathcal{V}$,若$S$为线性空间$V$的一个基,那么基于$S$的线性映射$T$的矩阵矩阵表示为$\mathcal{M}(T,S)$,若$S$为线性空间$V$的标准基,则简写为$\mathcal{M}(T)$
\end{definition}

接下来根据上述知识我们可以得到一些推论:

\begin{corollary}
	\label{cor:tri}
	若算子$T\in\mathcal{L}(\mathbb{C}^n)$则一定存在基$S$使得算子关于该基的矩阵$\mathcal{M}(T,S)$为上三角矩阵。
\end{corollary}

\begin{proof}
	为了证明任何线性算子 $ T \in \mathcal{L}(\mathbb{C}^n) $ 存在一个基 $ S $,使得其矩阵表示为上三角矩阵,我们可以使用数学归纳法:当 $ n=1 $ 时,任何 $ 1 \times 1 $ 矩阵本身就是上三角矩阵,结论显然成立。

	那么假设对于所有 $ n-1 $ 维复向量空间上的线性算子,存在一个基使得其矩阵为上三角矩阵。设 $ V $ 为 $ n $ 维复向量空间,$ T \in \mathcal{L}(V) $。由于 $ V $ 是复空间,$ T $ 必有特征值 $ \lambda_1 $,对应非零特征向量 $ v_1 $。令 $ U = \text{Span}(v_1) $,则 $ U $ 是 $ T $-不变的一维子空间。考虑线性映射 $ T - \lambda_1 I $,根据秩零化度定理:$$\dim V = \dim \text{null}(T - \lambda_1 I) + \dim \text{range}(T - \lambda_1 I)$$由于 $ \dim \text{null}(T - \lambda_1 I) \geq 1 $,得 $ \dim \text{range}(T - \lambda_1 I) \leq n - 1 $。选取 $ V $ 的补空间 $ W $ 使得 $ V = U \oplus W $。这里 $ W $ 是 $ n-1 $ 维子空间。定义投影 $ \pi: V \to W $,将任意 $ v \in V $ 分解为 $ v = u + w $(其中 $ u \in U, w \in W $),并令 $ \pi(v) = w $。定义线性算子 $ S: W \to W $ 为 $ S(w) = \pi(T(w)) $。由线性性保证,$ S $ 是 $ W $ 上的线性算子。根据归纳假设,存在 $ W $ 的一组基 $ v_2, \dots, v_n $,使得 $ S $ 在此基下的矩阵为上三角形式。即对任意 $ 2 \leq j \leq n $,有:$$S(v_j) \in \text{Span}(\{v_2, \dots, v_j\}).$$合并基 $ v_1, v_2, \dots, v_n $。对每个 $ v_j $($ j \geq 2 $),有:  
$$
T(v_j) = \pi(T(v_j)) + u_j \quad (u_j \in U).
$$
由于 $ \pi(T(v_j)) = S(v_j) \in \text{Span}(\{v_2, \dots, v_j\}) $,且 $ u_j \in \text{Span}(\{v_1\}) $,因此:  
$$
T(v_j) \in \text{Span}(\{v_1, v_2, \dots, v_j\}).
$$
对于 $ v_1 $,显然 $ T(v_1) = \lambda_1 v_1 \in \text{Span}(\{v_1\}) $。  
因此,基 $ \{v_1, v_2, \dots, v_n\} $ 使得 $ \mathcal{M}(T, S) $ 为上三角矩阵。

由数学归纳法,复向量空间上的任意线性算子均存在一组基,使其矩阵表示为上三角矩阵。  
即,存在基 $ S $ 使得 $ \mathcal{M}(T, S) $ 为上三角矩阵。
\begin{flushright}
		$\square$
	\end{flushright}
\end{proof}

\section{相似变换}

\subsection{相似矩阵}

首先给出矩阵相似的定义。

\begin{definition}{相似矩阵}
	若$n\times n$矩阵$\mathbf{A}$与矩阵$\mathbf{B}$相似,则存在 $n\times n$ 可逆矩阵$\mathbf{P}$使得$$\mathbf{A}=\mathbf{P}\mathbf{B}\mathbf{P}^{-1}$$记作$\mathbf{A}\sim \mathbf{B}$
\end{definition}

根据上述定义,我们不难发现如下的推论:

\begin{corollary}
	相似矩阵的性质:
	\begin{enumerate}
		\item 自反性:$\mathbf{A}$与$\mathbf{A}$相似;
		\item 对称性:若$\mathbf{A}$与$\mathbf{B}$相似,则$\mathbf{B}$与$\mathbf{A}$相似;
		\item 传递性:若$\mathbf{A}$与$\mathbf{B}$相似,$\mathbf{B}$与$\mathbf{C}$相似,则$\mathbf{A}$与$\mathbf{C}$相似
	\end{enumerate}
\end{corollary}

\begin{proof}
	证明留给读者,参见课后练习 B 组。
	\begin{flushright}
		$\square$
	\end{flushright}
\end{proof}

\begin{corollary}
	如果矩阵$\mathbf{A}\sim \mathbf{B}$,则它们的特征方程$p(\lambda)$相同,并满足$\text{Tr}\mathbf{A}=\text{Tr}\mathbf{B}$与$\det \mathbf{A}=\det \mathbf{B}$
\end{corollary}

\begin{proof}
	假设矩阵$\mathbf{A}$和$\mathbf{B}$相似,则存在可逆矩阵$\mathbf{P}$使得$\mathbf{B} = \mathbf{P}^{-1}\mathbf{A}\mathbf{P}$。

首先,证明它们的特征方程相同:

计算矩阵$\mathbf{B}$的特征方程多项式:
$$
\det(\lambda \mathbf{I} - \mathbf{B}) = \det(\lambda \mathbf{I} - \mathbf{P}^{-1}\mathbf{A}\mathbf{P})
$$
将$\lambda \mathbf{I}$改写为$\mathbf{P}^{-1}(\lambda \mathbf{I})\mathbf{P}$,因为$\mathbf{P}^{-1}(\lambda \mathbf{I})\mathbf{P} = \lambda \mathbf{I}$。于是:
$$
\det(\lambda \mathbf{I} - \mathbf{P}^{-1}\mathbf{A}\mathbf{P}) = \det(\mathbf{P}^{-1}(\lambda \mathbf{I} - \mathbf{A})\mathbf{P})
$$
利用行列式的乘法性质:
$$
\det(\mathbf{P}^{-1}(\lambda \mathbf{I} - \mathbf{A})\mathbf{P}) = \det(\mathbf{P}^{-1})\det(\lambda \mathbf{I} - \mathbf{A})\det(\mathbf{P})
$$
由于$\det(\mathbf{P}^{-1})\det(\mathbf{P}) = \det(\mathbf{P}^{-1}\mathbf{P}) = \det(\mathbf{I}) = 1$,因此:
$$
\det(\lambda \mathbf{I} - \mathbf{B}) = \det(\lambda \mathbf{I} - \mathbf{A})
$$
即$\mathbf{A}$和$\mathbf{B}$的特征方程相同。

接下来,由于特征方程相同,它们的特征方程多项式相同。特征方程多项式的形式为:
$$
\lambda^n - (\text{Tr}\mathbf{A})\lambda^{n-1} + \cdots + (-1)^n \det{\mathbf{A}}\footnote{为什么最后一项是$(-1)^n \det{\mathbf{A}}$?}
$$
比较多项式两边的对应系数:

第二项系数\footnote{根据推论\ref{cor:trace}}:$-\text{Tr}\mathbf{A} = -\text{Tr}\mathbf{B}$,因此$\text{Tr}\mathbf{A} = \text{Tr}\mathbf{B}$。常数项:$(-1)^n \det{\mathbf{A}} = (-1)^n \det{\mathbf{B}}$,因此$\det{\mathbf{A}} = \det{\mathbf{B}}$。

因此,矩阵$\mathbf{A}$和$\mathbf{B}$的迹和行列式相等。
\begin{flushright}
		$\square$
	\end{flushright}
\end{proof}

\begin{corollary}
	\label{cor:tri2}
	任意的$n$阶方阵$\mathbf{A}$都能相似于一个上三角矩阵。
\end{corollary}

\begin{proof}	
	设 $\mathbf{A}$ 是任意 $n \times n$ 复矩阵。$\mathbf{A}$ 可视为线性算子 $T_A: \mathbb{C}^n \to \mathbb{C}^n$ 在标准基 $B$ 下的矩阵表示,即 $\mathcal{M}(T_A, B) = \mathbf{A}$。  
	由推论\ref{cor:tri}证明,存在基 $S$ 使得 $\mathcal{M}(T_A, S)$ 是上三角矩阵。基变换矩阵 $\mathbf{P}$ 满足 $\mathbf{P}$ 的列是 $S$ 在基 $B$ 下的坐标向量,且  
	$$
	\mathcal{M}(T_A, S) = \mathbf{P}^{-1} \mathbf{A} \mathbf{P}.
	$$
	因此,$\mathbf{P}^{-1} \mathbf{A} \mathbf{P}$ 是上三角矩阵,即 $\mathbf{A}$ 相似于一个上三角矩阵。\begin{flushright}
		$\square$
	\end{flushright}
\end{proof}

\subsection{Hamilton-Cayley 定理}

根据上述的铺垫我们引入并证明 Hamilton-Cayley 定理

\begin{theorem}{Hamilton-Cayley 定理}
	多项式$P(\lambda)=\lambda^n+a_1\lambda^{n-1}+a_2\lambda^{n-2}+\cdots+a_n$为$n$阶方阵$\mathbf{A}$的特征方程,那么矩阵运算$$P(\mathbf{A}):=\mathbf{A}^n+a_1\mathbf{A}^{n-1}+a_2\mathbf{A}^{n-2}+\cdots+a_n\mathbf{I}=\mathbf{O}$$成立,$P(\mathbf{A})$为零矩阵$\mathbf{O}$。
\end{theorem}

这个定理挺奇妙,我们下面尝试一步步证明它。

\begin{proof}
	特征方程多项式将其可以改写为
	$$
   	P(\lambda) = (\lambda - \lambda_1)(\lambda - \lambda_2) \cdots (\lambda - \lambda_n)
   	$$根据推论\ref{cor:tri2},存在可逆矩阵$\mathbf{P}$,使得$$\mathbf{P}^{-1} \mathbf{A} \mathbf{P}=\begin{pmatrix}
 \lambda_1 & * & * & \cdots & *\\
  & \lambda_2 & * & \cdots & *\\
  &  & \lambda_3 & \cdots & *\\
  &  &  & \ddots & \vdots \\ 
  &  &  &  & \lambda_n
\end{pmatrix}$$将$\mathbf{P}^{-1} \mathbf{A} \mathbf{P}$代入矩阵运算即$$P(\mathbf{P}^{-1} \mathbf{A} \mathbf{P})=(\mathbf{P}^{-1} \mathbf{A} \mathbf{P} - \lambda_1\mathbf{I})(\mathbf{P}^{-1} \mathbf{A} \mathbf{P} - \lambda_2\mathbf{I}) \cdots (\mathbf{P}^{-1} \mathbf{A} \mathbf{P} - \lambda_n\mathbf{I})$$我们将其写成矩阵乘积的形式即\begin{tiny}
$$\begin{pmatrix}
 0 & * & * & \cdots & *\\
  & \lambda_2-\lambda_1 & * & \cdots & *\\
  &  & \lambda_3-\lambda_1 & \cdots & *\\
  &  &  & \ddots & \vdots \\ 
  &  &  &  & \lambda_n-\lambda_1
\end{pmatrix}\begin{pmatrix}
  \lambda_1-\lambda_2 & * & * & \cdots & *\\
  & 0 & * & \cdots & *\\
  &  & \lambda_3-\lambda_2 & \cdots & *\\
  &  &  & \ddots & \vdots \\ 
  &  &  &  & \lambda_n-\lambda_2
\end{pmatrix}\cdots\begin{pmatrix}
  \lambda_1-\lambda_n & * & * & \cdots & *\\
  & \lambda_2-\lambda_n & * & \cdots & *\\
  &  & \lambda_3-\lambda_n & \cdots & *\\
  &  &  & \ddots & \vdots \\ 
  &  &  &  & 0
\end{pmatrix}=\mathbf{O}$$
\end{tiny}由于单位矩阵$\mathbf{I}=\mathbf{P}\mathbf{P}^{-1}$可得$$P(\mathbf{P}^{-1} \mathbf{A} \mathbf{P})=(\mathbf{P}^{-1} \mathbf{A} \mathbf{P} - \lambda_1\mathbf{P}\mathbf{P}^{-1})(\mathbf{P}^{-1} \mathbf{A} \mathbf{P} - \lambda_2\mathbf{P}\mathbf{P}^{-1}) \cdots (\mathbf{P}^{-1} \mathbf{A} \mathbf{P} - \lambda_n\mathbf{P}\mathbf{P}^{-1})$$提出$\mathbf{P}\mathbf{P}^{-1}$部分可得$\mathbf{P}P(\mathbf{A})\mathbf{P}^{-1}=\mathbf{O}$即$P(\mathbf{A})=\mathbf{O}$
\begin{flushright}
		$\square$
	\end{flushright}
\end{proof}

Hamilton-Cayley 定理在很多方面也有应用,例如求矩阵多项式,求逆矩阵,伴随矩阵;下面我们就求逆矩阵讲讲如何利用 Hamilton-Cayley 定理求解。

设$n\times n$矩阵$\mathbf{A}$可逆,首先我们根据$n\times n$可逆矩阵的秩等于$n$,可得其特征方程多项式为$$P(\lambda)=\lambda^n+a_1\lambda^{n-1}+a_2\lambda^{n-2}+\cdots+a_n$$且$a_n=(-1)^n \det \mathbf{A}\neq 0$由Hamilton-Cayley定理,我们有等式$$\mathbf{A}^n+a_1\mathbf{A}^{n-1}+a_2\mathbf{A}^{n-2}+\cdots+a_n\mathbf{I}=\mathbf{O}$$化简这个式子我们可以得到$$-\frac{1}{a_n}\left( \mathbf{A}^n+a_1\mathbf{A}^{n-1}+a_2\mathbf{A}^{n-2}+\cdots a_{n-1}\mathbf{A} \right)=\mathbf{I}$$由此可得$$\mathbf{A}^{-1}=-\frac{1}{a_n}\left( \mathbf{A}^{n-1}+a_1\mathbf{A}^{n-2}+a_2\mathbf{A}^{n-3}+\cdots a_{n-1}\right)$$

\begin{example}
	使用 Hamilton-Cayley 定理计算方阵$\mathbf{A}=\begin{pmatrix}
 1 & 2 & 4\\
 0 & 1 & -1\\
 2 & 0 & 1
\end{pmatrix}$的逆$\mathbf{A}^{-1}$
\tcblower
\textcolor{purple}{\textbf{解}}: $\mathbf{A}$的特征方程多项式为$$P(\lambda)=\det \left( \lambda\mathbf{I}-\mathbf{A} \right)=-\lambda^3+3\lambda^2+5\lambda-11$$所以可得其逆矩阵$$\mathbf{A}^{-1}=\frac{1}{11}\left( -\mathbf{A}^2+3\mathbf{A}+5\mathbf{I} \right)=\frac{1}{11}\begin{pmatrix}
 1 & 2 & 6\\
 2 & 7 & -1\\
 2 & -4 & -1
\end{pmatrix}$$
\end{example}

\subsection{对角化矩阵}

首先给出对角矩阵的定义:

\begin{definition}{对角矩阵(Diagonal Matrix)与矩阵的对角化}
	若$n\times n$方阵$\mathbf{A}$的元素$a_{ij}$满足$a_{ij}=c_i\delta_{ij}$,其中$c_i\in \mathbb{F}$,$\delta_{ij}$为函数表示为$$\delta_{ij}=\left\{\begin{matrix} 
  1 \quad i=j \\  
  0 \quad i\neq j
\end{matrix}\right. \footnote{Kronecker Delta, Kronecker function, 克罗内克函数, https://mathworld.wolfram.com/KroneckerDelta.html}$$则该矩阵为对角矩阵,即该矩阵的元素除了其对角线上的元素以外均为 0,并记作$\text{diag}(a_{1},a_2,\cdots,a_n)$,即$$\text{diag}(a_{1},a_2,\cdots,a_n):=\begin{pmatrix}
 a_1 & 0 & \cdots &0 \\
 0 & a_2 & \cdots & 0 \\
 \vdots & \vdots & \ddots & \vdots\\
 0 & 0 & \cdots & a_1
\end{pmatrix}$$如果方阵$\mathbf{B}$与一个对角矩阵相似,则称$\mathbf{B}$称为可对角化。
\end{definition}

若方阵$\mathbf{A}\sim \text{diag}(a_{1},a_2,\cdots,a_n)$,则$\mathbf{A}$的特征方程$p(\lambda)=(\lambda-a_1)(\lambda-a_2)\cdots(\lambda-a_n)$因此这个对角矩阵的元素即为$\mathbf{A}$的特征值。
下面我们来描述矩阵可对角化的充分必要条件,若 $\mathbf{A}$ 可对角化,那么则有式子成立$$\mathbf{A}=\mathbf{P}\text{diag}(a_{1},a_2,\cdots,a_n)\mathbf{P}^{-1}$$设$\mathbf{P}$矩阵由列向量$\left\{ b_1,b_2,b_3,\cdots,b_n \right\}$构成,则$\mathbf{P}=\begin{pmatrix}
	b_1 & b_2 & b_3 & \cdots & b_n
\end{pmatrix}$由于$\mathbf{P}$可逆,所以其秩$r(\mathbf{P})=n$,因此$\left\{ b_1,b_2,b_3,\cdots,b_n \right\}$为其极大线性无关组,那么就可以写作下面的形式$$\mathbf{A}\begin{pmatrix}
	b_1 & b_2 & b_3 & \cdots & b_n
\end{pmatrix}=\begin{pmatrix}
	b_1 & b_2 & b_3 & \cdots & b_n
\end{pmatrix}\text{diag}(a_{1},a_2,\cdots,a_n)$$设矩阵 $\mathbf{B} = \begin{pmatrix} b_1 & b_2 & \cdots & b_n \end{pmatrix}$,其列由向量 $b_1, b_2, \ldots, b_n$ 组成。对角矩阵 $\mathbf{D} = \text{diag}(a_1, a_2, \cdots, a_n)$。则等式可写为:
$$
\mathbf{A} \mathbf{B} = \mathbf{B} \mathbf{D}
$$
\begin{itemize}
	\item 左边 $\mathbf{A} \mathbf{B}$ 的结果是一个矩阵,其第 $j$ 列为 $\mathbf{A} b_j$,即:
  $$
  \mathbf{A} \mathbf{B} = \begin{pmatrix} \mathbf{A} b_1 & \mathbf{A} b_2 & \cdots & \mathbf{A} b_n \end{pmatrix}
  $$
  	\item 右边 $\mathbf{B} \mathbf{D}$ 的结果是一个矩阵,其第 $j$ 列为 $\mathbf{B}$ 的第 $j$ 列乘以 $\mathbf{D}$ 的第 $j$ 个对角元素 $a_j$,即:
  $$
  \mathbf{B} \mathbf{D} = \begin{pmatrix} a_1 b_1 & a_2 b_2 & \cdots & a_n b_n \end{pmatrix}
  $$
\end{itemize}

因此,等式等价于:
$$
\begin{pmatrix} \mathbf{A} b_1 & \mathbf{A} b_2 & \cdots & \mathbf{A} b_n \end{pmatrix} = \begin{pmatrix} a_1 b_1 & a_2 b_2 & \cdots & a_n b_n \end{pmatrix}
$$
这表示对应列相等,即对于每个 $j = 1, 2, \ldots, n$:
$$
\mathbf{A} b_j = a_j b_j
$$

该等式化简后表示:每个列向量 $b_i$ 是矩阵 $\mathbf{A}$ 的特征向量,对应的特征值为 $a_i$。即:
$$
\mathbf{A} b_i = a_i b_i, \quad \forall i = 1, 2, \ldots, n
$$

因此我们可以得到矩阵$\mathbf{P}=\begin{pmatrix}
	b_1 & b_2 & b_3 & \cdots & b_n
\end{pmatrix}$的所有元素均为$\mathbf{A}$的特征向量排列形成的矩阵。

\begin{example}
	矩阵$\mathbf{A}=\begin{pmatrix}
			2 & 1 & 0\\
			0 & 3 & 0\\
			0 & 2 & 1
		\end{pmatrix}$,存在$\mathbf{A}=\mathbf{P}\mathbf{B}	\mathbf{P}^{-1}$且$\mathbf{B}$为对角矩阵,求矩阵$\mathbf{P}$的一个可能值。
	   	\tcblower
		\textcolor{purple}{\textbf{解}}: $\mathbf{A}$特征方程为$$\det (\lambda \mathbf{I}-\mathbf{A})=\begin{vmatrix}
			\lambda-2 & -1 & 0\\
			0 & \lambda-3 & 0\\
			0 & -2 & \lambda -1
		\end{vmatrix}=(\lambda -1)(\lambda -2)(\lambda -3)$$令其等于0可得$\mathbf{A}$有3个不同的特征值,分别为$1,2,3$。接下来分别代入3个特征值求方阵的特征向量:
		   	\begin{enumerate}
			\item 当 $\lambda=1$时$$\lambda \mathbf{I}-\mathbf{A}=\begin{pmatrix}
				-1 & -1 & 0\\
				0 & -2 & 0\\
				0 & -2 & 0
			\end{pmatrix}$$由此$(\lambda \mathbf{I}-\mathbf{A})v=\boldsymbol{0}$通过高斯消元法解线性方程组$$\begin{pmatrix}
				-1 & -1 & 0\\
				0 & -2 & 0\\
				0 & -2 & 0
			\end{pmatrix}\xrightarrow[r_3/2]{r_2/2} \begin{pmatrix}
				-1 & -1 & 0\\
				0 & -1 & 0\\
				0 & -1 & 0
			\end{pmatrix}\xrightarrow[r_3-r_2]{r_1-r_2}\begin{pmatrix}
				-1 & 0 & 0\\
				0 & -1 & 0\\
				0 & 0 & 0
			\end{pmatrix}$$还原为齐次方程为$$\left\{\begin{matrix} 
				-x_1 = 0 \\  
				-x_2 = 0
			\end{matrix}\right. $$ 其中 $x_3$ 为自由变量,所以$\lambda = 1$其一个特征向量为$(0,0,1)$
			\item 同理我们算出$\lambda =2$时其一个特征向量为$(1,0,0)$
			\item 同理我们算出$\lambda =3$时其一个特征向量为$(1,1,1)$
		   	\end{enumerate}
			以特征向量为列构造 $\mathbf{P}$,按特征值顺序排列:$$\mathbf{P} = \begin{pmatrix} v_1 & v_2 & v_3 \end{pmatrix} = \begin{pmatrix} 0 & 1 & 1 \\ 0 & 0 & 1 \\ 1 & 0 & 1 \end{pmatrix}$$计算行列式:$$\det(\mathbf{P}) = \det \begin{pmatrix} 0 & 1 & 1 \\ 0 & 0 & 1 \\ 1 & 0 & 1 \end{pmatrix} = 1 \cdot \det \begin{pmatrix} 1 & 1 \\ 0 & 1 \end{pmatrix} = 1 \cdot (1 \cdot 1 - 1 \cdot 0) = 1 \neq 0$$行列式非零,故 $\mathbf{P}$ 可逆。因此,矩阵 $\mathbf{P}$ 的一个可能值为:$$\mathbf{P}=\begin{pmatrix} 0 & 1 & 1 \\ 0 & 0 & 1 \\ 1 & 0 & 1 \end{pmatrix}$$
\end{example}

事实上,不一定每一个方阵都可以被对角化,下面给出可对角化方阵的充分条件:

\begin{theorem}{方阵可对角化的充分条件}
	\label{the:cfcon}
	如果$\mathbb{F}$上$n$阶方阵$\mathbf{A}$具有$n$个互不相同的特征值,则$\mathbf{A}$可以对角化。
\end{theorem}

\begin{proof}
	对每个特征值 $\lambda_i$,选取对应的特征向量 $v_i$,即 $\mathbf{A} v_i = \lambda_i v_i$,且 $v_i \neq \mathbf{0}$。考虑向量组 $\{v_1, v_2, \dots, v_n\}$。 使用数学归纳法:
	\begin{enumerate}
		\item 当 $k=1$ 时,单个特征向量 $v_1 \neq \mathbf{0}$,故线性无关。  
		\item 假设对 $k-1$ 个不同特征值的特征向量成立。考虑 $k$ 个不同特征值 $\lambda_1, \dots, \lambda_k$ 的特征向量 $v_1, \dots, v_k$。设线性组合:  
  $$
  c_1 v_1 + c_2 v_2 + \dots + c_k v_k = \mathbf{0} \quad (1)
  $$
  作用 $\mathbf{A}$ 得:  
  $$
  c_1 \lambda_1 v_1 + c_2 \lambda_2 v_2 + \dots + c_k \lambda_k v_k = \mathbf{0} \quad (2)
  $$
  将 $(1)$ 乘以 $\lambda_k$:  
  $$
  \lambda_k c_1 v_1 + \lambda_k c_2 v_2 + \dots + \lambda_k c_k v_k = \mathbf{0} \quad (3)
  $$
  $(2)$ 减 $(3)$:  
  $$
  c_1 (\lambda_1 - \lambda_k) v_1 + c_2 (\lambda_2 - \lambda_k) v_2 + \dots + c_{k-1} (\lambda_{k-1} - \lambda_k) v_{k-1} = \mathbf{0}
  $$
  由归纳假设,$v_1, \dots, v_{k-1}$ 线性无关,且 $\lambda_i \neq \lambda_k$(特征值互异),故 $c_i (\lambda_i - \lambda_k) = 0$ 推出 $c_i = 0$($i=1,2,\dots,k-1$)。代入 $(1)$ 得 $c_k v_k = \mathbf{0}$,因 $v_k \neq \mathbf{0}$,故 $c_k = 0$。因此,$v_1, \dots, v_k$ 线性无关。
	\end{enumerate}
	由归纳假设,$v_1, \dots, v_{k-1}$ 线性无关,且 $\lambda_i \neq \lambda_k$(特征值互异),故 $c_i (\lambda_i - \lambda_k) = 0$ 推出 $c_i = 0$($i=1,2,\dots,k-1$)。代入 $(1)$ 得 $c_k v_k = \mathbf{0}$,因 $v_k \neq \mathbf{0}$,故 $c_k = 0$。因此,$v_1, \dots, v_k$ 线性无关。

在本问题中,$k = n$,故 $\{v_1, v_2, \dots, v_n\}$ 线性无关。由于 $\mathbf{A}$ 是 $n$ 阶方阵,特征向量组构成 $\mathbb{F}^n$ 的一组基。

令 $\mathbf{P} = (v_1, v_2, \dots, v_n)$,则 $\mathbf{P}$ 可逆。计算:  
$$
\mathbf{A} \mathbf{P} = \mathbf{A} (v_1, v_2, \dots, v_n) = (\mathbf{A} v_1, \mathbf{A} v_2, \dots, \mathbf{A} v_n) = (\lambda_1 v_1, \lambda_2 v_2, \dots, \lambda_n v_n)
$$
令 $\mathbf{D} = \operatorname{diag}(\lambda_1, \lambda_2, \dots, \lambda_n)$,则:  
$$
\mathbf{P} \mathbf{D} = (v_1, v_2, \dots, v_n) \begin{pmatrix} \lambda_1 & & \\ & \ddots & \\ & & \lambda_n \end{pmatrix} = (\lambda_1 v_1, \lambda_2 v_2, \dots, \lambda_n v_n)
$$
故 $\mathbf{A} \mathbf{P} = \mathbf{P} \mathbf{D}$,即 $\mathbf{P}^{-1} \mathbf{A} \mathbf{P} = \mathbf{D}$,其中 $\mathbf{D}$ 为对角矩阵。

因此,$\mathbf{A}$ 可对角化。

\begin{flushright}
	$\square$
\end{flushright}
\end{proof}

\section{Jordan 标准型}

% \subsection{幂零变换}

% 首先我们明确一个概念,那就是线性映射算子$T\in \mathcal{L}(V)$可以表示为矩阵$\mathcal{M}(T)$,令$\mathbf{A}=\mathcal{M}(T)$我们可以得到向量$x\in \mathbb{F}^n$可得$T(x)=\mathbf{A}x$,每一次变换相当于在等式中左乘一个$\mathbf{A}$,由此我们可以推断:

% \begin{axiom}{矩阵乘公理}
% 	线性映射算子$T\in \mathcal{L}(V)$与向量$x\in \mathbb{F}^n$存在正整数$m \in \mathcal{N}^+$使得$T^m$表示$T$的$m$次复合,则$$T^m(x)=A^mx$$
% \end{axiom}

\subsection{Jordan 块}

在这里我们引入 Jordan\footnote{Jordan 法语读音 {\timesroman /ʒɔʁdɑ̃/} 音译为约旦,若尔当,英语读音 {\timesroman /'dʒɔ:dn/} 音译为乔丹,焦尔丹。实际上两种读法都有,但是由于 Jordan Marie Ennemond Camille 是法国人,所以我们中文常读作:``若尔当块''而非``乔丹块''} 块与 Jordan 标准型的概念,由于不是每一个方阵都可以相似变化为对角矩阵,但是可以相似为一个与之很近的形式。

\begin{definition}{Jordan 块(Jordan Block)}
	除对角线和超对角线外,其他位置均为零,对角线的每个元素均由单个数字 $\lambda$ 组成,对角线上方的每个元素均由 1 组成的$m$阶方阵为 Jordan 块,记作 $J_m(\lambda)$例如:
	$$
	J_m(\lambda):=\begin{pmatrix}
	\lambda & 1 &  &  & \\
	& \lambda & \ddots &  & \\
	&  & \ddots & 1 & \\
	&  &  & \lambda & 1\\
	&  &  &  &\lambda 
	\end{pmatrix}
	$$
	该矩阵的空余元素均为 0。此外$1\times 1$ 矩阵的退化情况被视为 Jordan 块,即使它缺少可填充 1 的上对角线。
\end{definition}

\subsection{代数重数}

首先我们举一个最简单的 Jordan 块;
$$
J_3(2)=\begin{pmatrix}
 2 & 1 & 0\\
 0 & 2 & 1\\
 0 & 0 & 2
\end{pmatrix}
$$
我们看看是否可以将其对角化,按照我们之前的过程,首先是计算该方阵的特征值,即计算:
$$
J_3(2) - \lambda \mathbf{I} = \begin{pmatrix} 2-\lambda & 1 & 0 \\ 0 & 2-\lambda & 1 \\ 0 & 0 & 2-\lambda \end{pmatrix}
$$
该矩阵为上三角矩阵,其行列式为对角线元素的乘积:$$\det(J_3(2) - \lambda \mathbf{I}) = (2-\lambda)^3$$最后解出$\lambda_{1,2,3}=2$;其次计算特征向量我们可以得到所有的特征向量均为$(1,0,0)$构成矩阵为$$\mathbf{P}=\begin{pmatrix}
 1 & 1 & 1\\
 0 & 0 & 0\\
 0 & 0 & 0
\end{pmatrix}$$由此可得$\mathbf{P}$不可逆,我们最后得到$J_3(2)$不可以被对角化。那这个时候读者可能会问,矩阵可对角化的充要条件是什么,我们上面的定理\ref{the:cfcon}只写到了充分条件,接下来我们引入代数重数和几何重数的例子。

首先作为特征方程多项式,展开后一定可以得到一个代数多项式并将其分解为
$$
P(\lambda) = (\lambda - \lambda_1)(\lambda - \lambda_2) \cdots (\lambda - \lambda_n)
$$
例如我们刚刚举的例子$J_3(2)$的特征方程多项式为$$\det(J_3(2) - \lambda \mathbf{I}) = (2-\lambda)^3$$那么我们称特征值2的代数重数为3。

\begin{definition}{代数重数}
	$n$阶方阵特征方程多项式最终可表示为$$P(\lambda)=\prod_{i=1}^{k} (\lambda-\lambda_i)^{t_i},\sum_{i=1}^{k}t_i=n$$且满足$\forall a\neq b\Longrightarrow\lambda_a\neq\lambda_b$则特征值$\lambda_i$的代数重数为$t_i$,记为$$m(\lambda_i):=t_i$$
\end{definition}

\subsection{几何重数与不变子空间}

那么几何重数则表示该特征值的几个特征向量的极大线性无关组的个数,也可以指的是这几个特征向量张成的空间的维数。接下里我们把``几个特征向量张成的空间''称作不变子空间。

\begin{definition}{不变子空间; 几何重数}
	设线性映射算子$T\in \mathcal{L}(V)$,$\lambda_0$为$T$的一个特征值,定义集合$V_{\lambda_0}$表示所有特征向量$x$可能值的集合,即:$$V_{\lambda_0}:=\left\{ x\mid T(x)=\lambda_0 x \right\}$$称为特征值$\lambda_0$的不变子空间;其几何重数为$\text{dim}\left( V_{\lambda_0} \right)$,记为$$g(\lambda_i):=\text{dim}\left( V_{\lambda_0} \right)$$
\end{definition}

关于代数重数与几何重数,有以下推论:

\begin{corollary}
	特征值的几何重数 $\leq$ 代数重数.
\end{corollary}

\begin{proof}
	我们不作严格证明,在计算特征方程的时候我们总是会得到一个未知数数量与式子数量均不超过 $n$ 的齐次线性方程组,所以其基础解系的维度不会超过 $n$,所以几何重数小于其代数重数。
	\begin{flushright}
		$\square$
	\end{flushright}
\end{proof}

接下来是一个重要的定理,它揭示了方阵可对角化的充要条件。

\begin{theorem}{方阵可对角化的充要条件}
	\label{the:candiag}
	矩阵可对角化等价于其所有特征值的代数重数等于几何重数。
\end{theorem}

\begin{proof}
	本定理不作严格证明,只做一个简洁的说明;首先可对角化 $\iff$ 存在 $n$ 个线性无关的特征向量,它们是构成 $V$ 的一组基,若所有 $m(\lambda) = g(\lambda)$,则特征向量总数 $\sum g(\lambda) = \sum m(\lambda) = n$,且不同特征值的特征向量线性无关,故它们构成基。并且若存在 $m(\lambda) > g(\lambda)$,则特征向量总数 $\sum g(\lambda) < n$,无法构成基,反映在$\mathbf{A}=\mathbf{P}\mathbf{B}\mathbf{P}^{-1}$中的矩阵$\mathbf{P}$不可逆,矩阵不可对角化。
	\begin{flushright}
		$\square$
	\end{flushright}
\end{proof}

\subsection{Jordan 标准型\footnote{本节选学}}

首先我们给出 Jordan 标准型的定义。

\begin{definition}{Jordan 标准型(Jordan canonical form)}
	Jordan 标准型是一种特殊类型的分块矩阵,其中每个分块由 Jordan 块组成,且每个 Jordan 分块的常数 $\lambda_i$ 可能不同,我们将其记作$\mathscr{J}$。具体而言,它是如下形式的分块矩阵:
	$$\mathscr{J}:=\begin{pmatrix}
		J_{m_1}(\lambda_1) & 0 & \cdots & 0\\
		0 & J_{m_2}(\lambda_2) & \cdots &0 \\
		\vdots & \vdots & \ddots & \vdots\\
		0 & 0 & \cdots &J_{m_i}(\lambda_i)
		\end{pmatrix}$$
\end{definition}

例如下面一个矩阵就是 Jordan 标准型(空白处均为0)$$\mathscr{J}=\left( \begin{array}{cccc|ccc|c}
 3 & 1 &  &  &  &  &  & \\
  & 3 & 1 &  &  &  &  & \\
  &  & 3 & 1 &  &  &  & \\
  &  &  & 3 &  &  &  & \\ \hline
  &  &  &  & 4 & 1 &  & \\
  &  &  &  &  & 4 & 1 & \\
  &  &  &  &  &  & 4 & \\ \hline
  &  &  &  &  &  &  & 5
\end{array} \right)=\begin{pmatrix}
 J_4(3) & 0 &0 \\
 0 & J_3(4) &0 \\
 0 & 0 & J_1(5)
\end{pmatrix}$$

通过 Jordan 标准型我们可以清晰地看出其各个特征值,以及它们的代数重数和几何重数,考虑 Jordan 标准型 $$\mathscr{J}=\begin{pmatrix}
		J_{m_1}(\lambda_1) & 0 & \cdots & 0\\
		0 & J_{m_2}(\lambda_2) & \cdots &0 \\
		\vdots & \vdots & \ddots & \vdots\\
		0 & 0 & \cdots &J_{m_i}(\lambda_i)
		\end{pmatrix}$$
\begin{itemize}
	\item $\mathscr{J}$ 的特征值由 Jordan 块的对角线元素给出,即 $\lambda_1, \lambda_2, \ldots, \lambda_i$(这些值可能重复);特征值的总数(考虑代数重数)为 $n = \sum_{k=1}^i m_k$,不同特征值的数量至多为 $i$(当所有 $\lambda_k$ 互异时取等号)。
	\item 对于每个特征值 $\lambda$,其代数重数为所有满足 $\lambda_k = \lambda$ 的 Jordan 块的大小之和,即 $\sum_{\{k \mid \lambda_k = \lambda\}} m_k$。
	\item 对于每个特征值 $\lambda$,其几何重数为所有满足 $\lambda_k = \lambda$ 的 Jordan 块的数量(即索引 $k$ 的个数)。
\end{itemize}
说的通俗一些就是对角元素的特征值$\lambda$,对角线上有几个相同的$\lambda$对应的特征值$\lambda$就是几重根,即为代数重数;对角元素是$\lambda$的 Jordan 块有几个,对应的线性无关的特征向量就有几个,即为几何重数;例如$\mathscr{J}=\begin{pmatrix}
 J_4(3) & 0 &0 \\
 0 & J_3(4) &0 \\
 0 & 0 & J_1(5)
\end{pmatrix}$,其特征值分别为$3,4,5$其中特征值 3 的几何重数为 1,代数重数为 4;特征值 4 的几何重数为 1,代数重数为 3;特征值 5 的几何重数为 1,代数重数为 1。根据定理\ref{the:candiag}我们可以得出该标准型不能被对角化。

\subsection{Jordan 标准型的幂与幂零算子}

首先我们看 Jordan 标准型的幂,由于相似变换不改变矩阵的幂(即若 $A = P\mathscr{J}P^{-1}$,则 $A^k = P\mathscr{J}^kP^{-1}$),研究 Jordan 标准型 $\mathscr{J}$ 的幂 $\mathscr{J}^k$ 等价于研究原矩阵 $A$ 的幂。而 $J$ 是由 Jordan 块 $J_m(\lambda)$ 沿对角线排列的分块对角矩阵:
$$
\mathscr{J} = \begin{pmatrix}
J_{m_1}(\lambda_1) & & \\
& J_{m_2}(\lambda_2) & \\
& & \ddots & \\
& & & J_{m_k}(\lambda_k)
\end{pmatrix}
$$因此,$\mathscr{J}^k$ 也是分块对角矩阵,其每个对角块即为对应 Jordan 块的幂:
$$
\mathscr{J}^k = \begin{pmatrix}
J_{m_1}^k(\lambda_1) & & \\
& J_{m_2}^k(\lambda_2) & \\
& & \ddots & \\
& & & J_{m_k}^k(\lambda_k)
\end{pmatrix}
$$问题归结为:如何计算一个 Jordan 块 $J_m(\lambda)$ 的 $k$ 次幂。

一个 $m$ 阶 Jordan 块 $J_m(\lambda)$ 可以分解为两部分之和:
$$
J_m(\lambda) = \lambda I_m + N_m
$$其中\footnote{实际上下面的$N_m$就是幂零矩阵}$$
    N_m = \begin{pmatrix}
    0 & 1 & & \\
    & 0 & \ddots & \\
    & & \ddots & 1 \\
    & & & 0
    \end{pmatrix}
    $$
利用二项式定理(因为 $\lambda I_m$ 与 $N_m$ 是可交换(见课后习题)的:$(\lambda I_m) N_m = N_m (\lambda I_m)$),我们可以计算 $J_m^k(\lambda)$:
$$
(J_m(\lambda))^k = (\lambda I_m + N_m)^k = \sum_{r=0}^{k} C_k^r (\lambda I_m)^{k-r} (N_m)^r = \sum_{r=0}^{k} C_k^r \lambda^{k-r} (N_m)^r
$$
其中 $\displaystyle C_k^r = \frac{k!}{r!(k-r)!}$ 是二项式系数。接下来我们先来研究$N_m$的特性,在此之前我们先讲一下幂零算子以及它的矩阵表示。

首先我们明确一个概念,那就是线性映射算子$T\in \mathcal{L}(\mathbb{F}^n)$可以表示为矩阵$\mathcal{M}(T)$,令$\mathbf{A}=\mathcal{M}(T)$我们可以得到向量$x\in \mathbb{F}^n$可得$T(x)=\mathbf{A}x$,每一次变换相当于在等式中左乘一个$\mathbf{A}$,由此我们可以推断:

\begin{axiom}{矩阵乘公理}
	线性映射算子$T\in \mathcal{L}(\mathbb{F}^n)$与向量$x\in \mathbb{F}^n$存在正整数$m \in \mathcal{N}^+$使得$T^m$表示$T$的$m$次复合,则$$T^m(x)=A^mx$$
\end{axiom}

接下来我们引入幂零算子的概念。

\begin{definition}{幂零算子(Nilpotent operator)}
	线性映射算子$T\in \mathcal{L}(\mathbb{F}^n)$存在$m$次复合使得$\forall x\in \mathbb{F}^n,x\neq \boldsymbol{0}$转化为零向量即$T^m(x)=\boldsymbol{0}$,则$T$为幂零算子;而其对应的矩阵$\mathcal{M}(T)$为幂零矩阵。
\end{definition}

根据上面的定义我们可以得出 $N_m$ 是幂零矩阵,具体来说(请读者计算):\begin{itemize}
	\item  $(N_m)^1$:次对角线(紧邻主对角线上方的那条对角线)上的元素为 1,其余位置为 0。
	\item $(N_m)^2$:次次对角线(主对角线向上数第二条)上的元素为 1,其余位置为 0;
	\item $\cdots$
	\item $(N_m)^{m-1}$:仅右上角元素为 1,其余位置为 0。
	\item $(N_m)^m = 0$:零矩阵。
\end{itemize}

由于 $N_m$ 是幂零指数为 $m$ 的幂零算子,当 $r \geq m$ 时,$(N_m)^r = 0$。因此,上述求和实际上只需从 $r=0$ 到 $r=\min(k, m-1)$:
$$
(J_m(\lambda))^k = \sum_{r=0}^{\min(k, m-1)} C_k^r \lambda^{k-r} (N_m)^r
$$

\begin{example}
3 阶 Jordan 块$J_3(\lambda)$与其幂零矩阵 $N_3$ 为
$$
J_3(\lambda) = \begin{pmatrix}
\lambda & 1 & 0 \\
0 & \lambda & 1 \\
0 & 0 & \lambda
\end{pmatrix}, \quad N_3 = \begin{pmatrix}
0 & 1 & 0 \\
0 & 0 & 1 \\
0 & 0 & 0
\end{pmatrix}
$$
计算 $(J_3(\lambda))^k$:
\tcblower
\textcolor{purple}{\textbf{解}}: 
$$
(J_3(\lambda))^k = \sum_{r=0}^{\min(k, 2)} C_k^r \lambda^{k-r} (N_3)^r
$$
其中:
\begin{itemize}
	\item $(N_3)^0 = I_3 = \begin{pmatrix} 1 & 0 & 0 \\ 0 & 1 & 0 \\ 0 & 0 & 1 \end{pmatrix}$
	\item $(N_3)^1 = N_3 = \begin{pmatrix} 0 & 1 & 0 \\ 0 & 0 & 1 \\ 0 & 0 & 0 \end{pmatrix}$
	\item $(N_3)^2 = \begin{pmatrix} 0 & 0 & 1 \\ 0 & 0 & 0 \\ 0 & 0 & 0 \end{pmatrix}$
	\item $(N_3)^r = 0,\forall r \geq 3$.
\end{itemize}
因此:
$$
(J_3(\lambda))^k = C_k^0 \lambda^{k} (N_3)^0 + C_k^1 \lambda^{k-1} (N_3)^1 + C_k^2\lambda^{k-2} (N_3)^2
$$
$$
= \lambda^k \begin{pmatrix} 1 & 0 & 0 \\ 0 & 1 & 0 \\ 0 & 0 & 1 \end{pmatrix} + k\lambda^{k-1} \begin{pmatrix} 0 & 1 & 0 \\ 0 & 0 & 1 \\ 0 & 0 & 0 \end{pmatrix} + \frac{k(k-1)}{2}\lambda^{k-2} \begin{pmatrix} 0 & 0 & 1 \\ 0 & 0 & 0 \\ 0 & 0 & 0 \end{pmatrix}
$$
$$
= \begin{pmatrix}
\lambda^k & k\lambda^{k-1} & \frac{k(k-1)}{2}\lambda^{k-2} \\
0 & \lambda^k & k\lambda^{k-1} \\
0 & 0 & \lambda^k
\end{pmatrix}
$$
\end{example}
下面做一个总结:主对角线全为 $\lambda^k$,第一条上对角线全为 $k\lambda^{k-1}$,第二条上对角线全为 $\frac{k(k-1)}{2}\lambda^{k-2}$。当 $k \geq 2$ 时,结构稳定;当 $k=1$ 时,最后一项消失,退回到 $J_3(\lambda)$;当 $k=0$ 时,为单位阵。

\subsection{将矩阵相似化为 Jordan 标准型}

虽然有些矩阵很特殊,不能将其化为对角矩阵,但是我们可以将其化为 Jordan 标准型。首先我们先不讲到底哪些矩阵可以化为 Jordan 标准型\footnote{事实上后面会提及所有方阵均可化为 Jordan 标准型},先讲如何把矩阵化为 Jordan 标准型。

考虑从如下的 Jordan 标准形 $$\mathscr{J}=\left( \begin{array}{cccc|ccc|c}
 3 & 1 &  &  &  &  &  & \\
  & 3 & 1 &  &  &  &  & \\
  &  & 3 & 1 &  &  &  & \\
  &  &  & 3 &  &  &  & \\ \hline
  &  &  &  & 4 & 1 &  & \\
  &  &  &  &  & 4 & 1 & \\
  &  &  &  &  &  & 4 & \\ \hline
  &  &  &  &  &  &  & 5
\end{array} \right)$$ 出发,即存在矩阵$\mathbf{A}$满足$\mathbf{P}^{-1}\mathbf{A}\mathbf{P}=\mathscr{J}$,通过等式两侧左乘一个矩阵$\mathbf{P}$得到$$\mathbf{A}\mathbf{P}=\mathbf{P}\mathscr{J}$$设矩阵的列向量构成为$\begin{pmatrix}
 p_1 & p_2 & p_3 & \cdots &p_8
\end{pmatrix}$,根据每列的值,我们依次得出:\begin{equation}
	\label{eq:simToJordan}
	\begin{aligned}
    \mathbf{A}p_1&= 3p_1\\
    \mathbf{A}p_2&= p_1+3p_2\\
    \mathbf{A}p_3&= p_2+3p_3\\
    \mathbf{A}p_4&= p_3+3p_4\\ \hline
    \mathbf{A}p_5&= 4p_5\\
    \mathbf{A}p_6&= p_5+4p_6\\
    \mathbf{A}p_7&= p_6+4p_7\\ \hline
    \mathbf{A}p_8&= 5p_8\\
\end{aligned}\quad \Longrightarrow \quad \begin{aligned}
    (\mathbf{A}-3\mathbf{I})p_1&= \boldsymbol{0}\\
    (\mathbf{A}-3\mathbf{I})p_2&= p_1\\
    (\mathbf{A}-3\mathbf{I})p_3&= p_2\\
    (\mathbf{A}-3\mathbf{I})p_4&= p_3\\ \hline
    (\mathbf{A}-4\mathbf{I})p_5&= \boldsymbol{0}\\
    (\mathbf{A}-4\mathbf{I})p_6&= p_5\\
    (\mathbf{A}-4\mathbf{I})p_7&= p_6\\ \hline
    (\mathbf{A}-5\mathbf{I})p_8&= \boldsymbol{0}\\
\end{aligned}
\end{equation}

矩阵的列向量$p_1,p_5,p_8$都是矩阵$\mathbf{A}$的特征向量,对应的特征值分别为$3,4,5$,对应的代数重数分别为$4,3,1$,实际上这些关系揭示了 Jordan 标准型的核心结构:每个 Jordan 块对应一个由特征向量和广义特征向量组成的链(称为 \textbf{Jordan 链})。具体来说就是如下定义:

\begin{definition}{广义特征向量(Generalized Eigenvector)}
	$n\times n$ 矩阵 $\mathbf{A}$ 的广义特征向量是向量 $v$ 满足$$(\mathbf{A}-\lambda\mathbf{I})^k v=\boldsymbol{0}$$
\end{definition}

\begin{itemize}
	\item 对于特征值 $\lambda=3$,存在长度为 4 的 Jordan 链:
	\begin{itemize}
		\item $p_1$ 是特征向量(满足 $(\mathbf{A}-3\mathbf{I})p_1=\boldsymbol{0}$)
		\item $p_2$ 是一阶广义特征向量(满足 $(\mathbf{A}-3\mathbf{I})p_2=p_1$)
		\item $p_3$ 是二阶广义特征向量(满足 $(\mathbf{A}-3\mathbf{I})p_3=p_2$)
		\item $p_4$ 是三阶广义特征向量(满足 $(\mathbf{A}-3\mathbf{I})p_4=p_3$)
	\end{itemize}
	\item 类似地,$\lambda=4$ 对应长度为 3 的链:$p_5$(特征向量)$\rightarrow$ $p_6$(一阶广义)$\rightarrow$ $p_7$(二阶广义)
	\item $\lambda=5$ 对应长度为 1 的链:仅特征向量 $p_8$。
\end{itemize}

给定矩阵 $\mathbf{A}$,求其 Jordan 标准型的步骤如下:
% \begin{enumerate}
%   \item \textbf{计算特征值与代数重数}:求解特征方程 $\det(\mathbf{A}-\lambda\mathbf{I})=0$,得到特征值 $\lambda_i$ 及其代数重数 $m_i$。
%   \item \textbf{对每个特征值 $\lambda_i$ 计算广义特征空间}:
%     \begin{itemize}
%       \item 定义广义特征空间:$W_k(\lambda_i) = \text{null}\left((\mathbf{A}-\lambda_i\mathbf{I})^k\right)$
%       \item 计算幂零指数 $\rho_i$:满足 $W_{\rho_i}(\lambda_i) = W_{\rho_i+1}(\lambda_i)$ 的最小整数(即空间稳定化的最小幂次)
%     \end{itemize}
%   \item \textbf{构造 Jordan 链}:
%     \begin{enumerate}
%       \item 从最高阶广义特征向量开始:选 $v_{\rho_i} \in W_{\rho_i}(\lambda_i) \setminus W_{\rho_i-1}(\lambda_i)$
%       \item 逆向生成链:
%         \begin{align*}
%           v_{\rho_i-1} &= (\mathbf{A}-\lambda_i\mathbf{I})v_{\rho_i} \\
%           v_{\rho_i-2} &= (\mathbf{A}-\lambda_i\mathbf{I})v_{\rho_i-1} \\
%           &\vdots \\
%           v_1 &= (\mathbf{A}-\lambda_i\mathbf{I})v_2
%         \end{align*}
%         其中 $v_1$ 是特征向量,整个链 $\{v_1, v_2, \dots, v_{\rho_i}\}$ 线性无关。
%       \item 若广义特征空间未满(即 $\dim W_{\rho_i}(\lambda_i) > \rho_i$),重复选取新向量生成新链。
%     \end{enumerate}
%   \item \textbf{组合变换矩阵}:将所有 Jordan 链的向量按顺序排列为 $\mathbf{P}$ 的列向量。
%   \item \textbf{得到 Jordan 标准型}:$\mathscr{J} = \mathbf{P}^{-1}\mathbf{A}\mathbf{P}$。
% \end{enumerate}

\begin{enumerate}
	\item 求解特征方程 $\det(\mathbf{A}-\lambda\mathbf{I})=0$,得到特征值 $\lambda_i$ 及其代数重数 $m_i$。
	\item 对每个特征值 $\lambda_i$ 计算广义特征空间,即
	\begin{itemize}
		\item 定义广义特征空间:$W_k(\lambda_i) = \text{null}\left((\mathbf{A}-\lambda_i\mathbf{I})^k\right)$
		\item 计算幂零指数 $\rho_i$:满足 $W_{\rho_i}(\lambda_i) = W_{\rho_i+1}(\lambda_i)$ 的最小整数(即空间稳定化的最小幂次)
	\end{itemize}
	\item 构造 Jordan 链,从最高阶广义特征向量开始:选 $v_{\rho_i} \in W_{\rho_i}(\lambda_i) \setminus W_{\rho_i-1}(\lambda_i)\footnote{符号$A \setminus B$表示减集合,即$\left\{ x \mid x\in A ~ \text{and} ~ x\notin B \right\}$,事实上线性代数还有一种空间集合运算,称为商空间,用的是$A / B$表示。}$
	\item 逆向生成 Jordan 链
        \begin{align*}
          v_{\rho_i-1} &= (\mathbf{A}-\lambda_i\mathbf{I})v_{\rho_i} \\
          v_{\rho_i-2} &= (\mathbf{A}-\lambda_i\mathbf{I})v_{\rho_i-1} \\
          &\vdots \\
          v_1 &= (\mathbf{A}-\lambda_i\mathbf{I})v_2
        \end{align*}
        其中 $v_1$ 是特征向量,整个链 $\{v_1, v_2, \dots, v_{\rho_i}\}$ 线性无关。
	\item 若广义特征空间未满(即 $\dim W_{\rho_i}(\lambda_i) > \rho_i$),重复选取新向量生成新链。
	\item 组合变换矩阵,将所有 Jordan 链的向量按顺序排列为 $\mathbf{P}$ 的列向量。
	\item 得到 Jordan 标准型$\mathscr{J} = \mathbf{P}^{-1}\mathbf{A}\mathbf{P}$。
\end{enumerate}

我们来详细看一个例子:

\begin{example}
	矩阵$\mathbf{A}=\begin{pmatrix}
 -12 & 1 & -7 & 8\\
 -27 & 3 & -11 & 14\\
 -41 & 2 & -16 & 22\\
 -56 & 3 & -26 & 33
\end{pmatrix}$存在矩阵$\mathbf{P}$与Jordan标准形$\mathscr{J}$使得$\mathscr{J}=\mathbf{P}^{-1}\mathbf{A}\mathbf{P}$成立,求$\mathbf{P}$的一个可能值。

\tcblower
\textcolor{purple}{\textbf{解}}: 

为了保证有序性,使用红色文本代表选取此Jordan链作为$\mathbf{P}$的列向量,且保证列向量线性无关。蓝色文本表示当前条件下可选择另一个基础解系生成的Jordan链,两者取其一即可。

\vspace{1em}

首先根据特征向量 $v$ 对应的特征值 $\lambda$ 的定义,我们有$$(\mathbf{A}-\lambda\mathbf{I})v=\boldsymbol{0}$$仅有行列式$\det (\mathbf{A}-\lambda\mathbf{I})=0$特征多项式$$P(\lambda)=\begin{vmatrix}
 -12-\lambda  & 1 & -7 & 8\\
 -27 & 3-\lambda  & -11 & 14\\
 -41 & 2 & -16-\lambda  & 22\\
 -56 & 3 & -26 & 33-\lambda 
\end{vmatrix}=\lambda ^4-8\lambda^3+23\lambda^2-28\lambda+12$$化简为$$P(\lambda)=(\lambda-1)(\lambda-2)(\lambda-2)(\lambda-3)=0$$我们得到三个特征值为$\lambda_1=1,\lambda_{2,3}=2,\lambda_4=3$,接下来为每个特征值求线性独立的广义特征向量,因为$\mathbf{A}-\lambda\mathbf{I}$是幂零矩阵我们需要先确定其最大的秩。

\vspace{1em}

首先考虑$\lambda=1$时其代数重数为$1$;
\begin{enumerate}
	\item 确定广义特征向量的最大秩 $$r\left( \left( \begin{pmatrix}
 -12 & 1 & -7 & 8\\
 -27 & 3 & -11 & 14\\
 -41 & 2 & -16 & 22\\
 -56 & 3 & -26 & 33
\end{pmatrix}-1\mathbf{I} \right)^1 \right)=3$$
	\item 求其 Jordan 链,即求每一个矩阵幂下的关于向量$x$的基础解系;当幂为1时,其秩为3,关于$x$矩阵乘$$\left( \begin{pmatrix}
 -12 & 1 & -7 & 8\\
 -27 & 3 & -11 & 14\\
 -41 & 2 & -16 & 22\\
 -56 & 3 & -26 & 33
\end{pmatrix}-1\mathbf{I} \right)^1x=\boldsymbol{0}$$的一个基础解系$\mathcal{B}$为$\mathcal{B}=\left\{ (1,2,3,4) \right\}$\footnote{如果读者不知道这一部分是在做什么,请复习章节\ref{sec:solSpace}},带入此基础解系$\mathcal{B}=\left\{ (1,2,3,4) \right\}$
	\begin{enumerate}
	\item 令$x_1=(1,2,3,4)$可得$$\left( \begin{pmatrix}
 -12 & 1 & -7 & 8\\
 -27 & 3 & -11 & 14\\
 -41 & 2 & -16 & 22\\
 -56 & 3 & -26 & 33
\end{pmatrix}-1\mathbf{I} \right)^{(1-1)}x_1=\mathbf{I}x_1\neq \boldsymbol{0}$$所以该向量仅为特征向量,为此广义特征向量生成 Jordan 链为$\textcolor{red}{v_1}=(1,2,3,4)$
	\end{enumerate}
\end{enumerate}

\vspace{1em}

接下来考虑$\lambda=2$时其代数重数为$2$;
\begin{enumerate}
	\item 确定广义特征向量的最大秩 $$r\left( \left( \begin{pmatrix}
	-12 & 1 & -7 & 8\\
	-27 & 3 & -11 & 14\\
	-41 & 2 & -16 & 22\\
	-56 & 3 & -26 & 33
	\end{pmatrix}-2\mathbf{I} \right)^1 \right)=3,r\left( \left( \begin{pmatrix}
	-12 & 1 & -7 & 8\\
	-27 & 3 & -11 & 14\\
	-41 & 2 & -16 & 22\\
	-56 & 3 & -26 & 33
	\end{pmatrix}-2\mathbf{I} \right)^2 \right)=2$$
	\item 求其 Jordan 链,当幂为1时其秩为3,当幂为2时其秩为2,从最高次幂构建Jordan链,关于$x$的矩阵乘法$$\left( \begin{pmatrix}
	-12 & 1 & -7 & 8\\
	-27 & 3 & -11 & 14\\
	-41 & 2 & -16 & 22\\
	-56 & 3 & -26 & 33
	\end{pmatrix}-2\mathbf{I} \right)^2x=\boldsymbol{0}$$的一个基础解系$\mathcal{B}$为$\mathcal{B}=\left\{ (-1,-1,1,0),(1,4,0,3) \right\}$\begin{enumerate}
		\item 令$x_1=(-1,-1,1,0)$,$$\left( \begin{pmatrix}
	-12 & 1 & -7 & 8\\
	-27 & 3 & -11 & 14\\
	-41 & 2 & -16 & 22\\
	-56 & 3 & -26 & 33
	\end{pmatrix}-2\mathbf{I} \right)^{(2-1)}x_1\neq \boldsymbol{0}$$所以该向量仅为一阶的广义特征向量,为此广义特征向量生成 Jordan 链为$$\textcolor{red}{v_1}=(-1,-1,1,0)\leftarrow \textcolor{red}{v_2}=\left( \begin{pmatrix}
	-12 & 1 & -7 & 8\\
	-27 & 3 & -11 & 14\\
	-41 & 2 & -16 & 22\\
	-56 & 3 & -26 & 33
	\end{pmatrix}-2\mathbf{I} \right)^1v_1=(6,15,21,27)$$

		\item 令$x_2=(1,4,0,3)$,$$\left( \begin{pmatrix}
	-12 & 1 & -7 & 8\\
	-27 & 3 & -11 & 14\\
	-41 & 2 & -16 & 22\\
	-56 & 3 & -26 & 33
	\end{pmatrix}-2\mathbf{I} \right)^{(2-1)}x_1\neq \boldsymbol{0}$$所以该向量仅为一阶的广义特征向量,为此广义特征向量生成 Jordan 链为$$\textcolor{blue}{v_1}=(1,4,0,3)\leftarrow \textcolor{blue}{v_2}=\left( \begin{pmatrix}
	-12 & 1 & -7 & 8\\
	-27 & 3 & -11 & 14\\
	-41 & 2 & -16 & 22\\
	-56 & 3 & -26 & 33
	\end{pmatrix}-2\mathbf{I} \right)^1v_1=(-14,-35,-49,-63)$$
	\end{enumerate}
\end{enumerate}

\vspace{1em}

接下来考虑$\lambda=3$时其代数重数为$1$;

\begin{enumerate}
	\item 确定广义特征向量的最大秩$$r\left( \left( \begin{pmatrix}
	-12 & 1 & -7 & 8\\
	-27 & 3 & -11 & 14\\
	-41 & 2 & -16 & 22\\
	-56 & 3 & -26 & 33
	\end{pmatrix}-3\mathbf{I} \right)^1 \right)=3$$当幂为1时,其秩为3,关于$x$矩阵乘$$\left( \begin{pmatrix}
	-12 & 1 & -7 & 8\\
	-27 & 3 & -11 & 14\\
	-41 & 2 & -16 & 22\\
	-56 & 3 & -26 & 33
	\end{pmatrix}-3\mathbf{I} \right)^1x=\boldsymbol{0}$$的一个基础解系$\mathcal{B}$为$\mathcal{B}=\left\{ (2,2,4,7) \right\}$
	\begin{enumerate}
		\item 令$x_1=(2,2,4,7)$可得$$\left( \begin{pmatrix}
	-12 & 1 & -7 & 8\\
	-27 & 3 & -11 & 14\\
	-41 & 2 & -16 & 22\\
	-56 & 3 & -26 & 33
	\end{pmatrix}-3\mathbf{I} \right)^{(1-1)}x_1=\mathbf{I}x_1\neq \boldsymbol{0}$$所以该向量仅为特征向量,为此广义特征向量生成 Jordan 链为$\textcolor{red}{v_1}=(2,2,4,7)$
	\end{enumerate}
\end{enumerate}
\vspace{1em}

最后将选取列的几个 Jordan 链为$$\textcolor{red}{\left( 1,2,3,4 \right),\left( 6,15,21,27 \right) \rightarrow \left( -1,-1,1,0 \right),\left( 2,2,4,7 \right)}$$矩阵列是来自顺序排列所选链的广义特征向量为$$\mathbf{P}=\begin{pmatrix}
	1 & 6 & -1 & 2\\
	2 & 15 & -1 & 2\\
	3 & 21 & 1 & 4\\
	4 & 27 & 0 & 7
	\end{pmatrix}$$
\end{example}

讲完了如何化为 Jordan 标准形,最后我们来看看一个推论:

\begin{corollary}
	任何$\mathbb{C}^{n\times n}$矩阵$\mathbf{A}$均存在矩阵$\mathbf{P}$和Jordan标准形$\mathscr{J}$使得$$\mathscr{J}=\mathbf{P}^{-1}\mathbf{A}\mathbf{P}$$成立
\end{corollary}

\begin{proof}
	我们这里不做严格证明,首先我们选取广义特征向量的时候都会选取一个 Jordan 链,不同于对角矩阵,Jordan标准形扩展了特征向量的意义,使得存在一个广义特征向量使得其能够化为一个个 Jordan 块,保证$\mathbf{P}$线性无关,由此任何方阵均存在一个$\mathscr{J}$与其相似。
\end{proof}

\section{章节练习}

\subsection{A组}

\begin{reidai}
	若算子$T\in \mathcal{L}(\mathbb{C}^3)$定义为$T((x,y,z))=(3x+z,y-x,x+y+z)$,求$T$的特征值以及对应的特征向量。
\end{reidai}

\begin{reidai}
	已知 12 是矩阵$$\mathbf{A}=\begin{pmatrix}
	7 & 4 & -1\\
	4 & 7 & -1\\
	-4 & a & 4
	\end{pmatrix}$$的特征值,求$a$与$\text{Tr}\mathbf{A}$。
\end{reidai}

\begin{reidai}
	设矩阵$\mathbf{A}=\begin{pmatrix}
		3 & 1 & 0 \\
		0 & 2 & 0 \\
		0 & -1 & 1
		\end{pmatrix}^{100}$,求$\mathbf{A}^{-1}$和$\mathbf{A}^{100}$。
\end{reidai}

\begin{reidai}
	已知矩阵$\mathbf{A}=\begin{pmatrix}
	-2 & 0 & 0 \\
	2 & x & 2 \\
	3 & 1 & 1
	\end{pmatrix}$相似于对角矩阵$\mathbf{B}=\text{diag}\left( -1,2,y \right)$,求$x,y$的值。
\end{reidai}