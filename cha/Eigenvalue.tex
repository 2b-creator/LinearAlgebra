\chapter{特征值理论}
\begin{center}
	% \textcolor[RGB]{255, 0, 0}{\faHeart}所以生命啊,它苦涩如歌.\textcolor[RGB]{255, 0, 0}{\faHeart}
	「人生若只如初见,何事秋风悲画扇」
\end{center}
\rightline{——《木兰花$\cdot$令拟古决绝词》}
\vspace{-5pt}
\begin{center}
	\pgfornament[width=0.36\linewidth,color=lsp]{88}
\end{center}

\section{特征值与特征向量}

\subsection{特征值}

首先我们来定义特征值:

\begin{definition}{特征值(Eigenvalue)}
	若存在向量$v\in V$,线性映射算子$T\in \mathcal{L}(V)$与标量$\lambda \in \mathbb{F}$满足$$T(v)=\lambda v$$则称$\lambda$为$T$的特征值。
\end{definition}

下面是一个例子:

\begin{example}
	若算子$T\in \mathcal{L}(\mathbb{F}^2)$定义为$T((x,y))=(-y,x)$求;

	(1)算子$T$以标准基进行变换的矩阵表示$\mathcal{M}(T)$;

	(2)当$\mathbb{F}=\mathbb{R}$时,验证$T$是否有特征值,如果有则求之,否则说明理由;

	(3)当$\mathbb{F}=\mathbb{C}$时,验证$T$是否有特征值,如果有则求之,否则说明理由。

	\tcblower
	\textcolor{purple}{\textbf{解}}: 
	
	(1) 算子 $ T $ 在标准基下的矩阵表示:

	标准基为 $ e_1 = (1, 0) $ 和 $ e_2 = (0, 1) $。计算 $ T $ 作用在基向量上的结果:
	\begin{itemize}
		\item $ T(e_1) = T((1, 0)) = (0, 1) = e_2 $
		\item $ T(e_2) = T((0, 1)) = (-1, 0) = -e_1 $
	\end{itemize}
	
	因此,矩阵表示为:
	$$
	\mathcal{M}(T) = \begin{pmatrix} 0 & -1 \\ 1 & 0 \end{pmatrix}
	$$
	所以答案是:
	$$
	\begin{pmatrix} 0 & -1 \\ 1 & 0 \end{pmatrix}
	$$

	(2) 假设存在实数$\lambda$和非零向量$(x,y)\in\mathbb{R}^2$,使得:
	$$
	T(x,y) = \lambda(x,y) \implies (-y, x) = (\lambda x, \lambda y)
	$$
	这等价于方程组:
	$$
	\begin{cases}
	-y = \lambda x \\
	x = \lambda y
	\end{cases}
	$$
	\begin{enumerate}
		\item 从第一式得 $y = -\lambda x$,代入第二式:
		$$
		x = \lambda (-\lambda x) = -\lambda^2 x
		$$
		\item 整理得 $x(1 + \lambda^2) = 0$。
		\begin{itemize}
			\item 若$x \neq 0$,则 $1 + \lambda^2 = 0$。但在$\mathbb{R}$中无解。 
			\item  若$x = 0$,则第一式给出$y = 0$,与$v$非零矛盾。 
		\end{itemize}
		所以在实数域$\mathbb{R}$中,$T$没有特征值

		\item 假设存在复数$\lambda$和非零向量$(x,y)\in\mathbb{C}^2$,使得:
		$$
		T(x,y) = \lambda(x,y) \implies (-y, x) = (\lambda x, \lambda y)
		$$
		方程组仍为:
		$$
		\begin{cases}
		-y = \lambda x \\
		x = \lambda y
		\end{cases}
		$$
		\begin{enumerate}
			\item 从第一式得 $y = -\lambda x$,代入第二式:$$x = \lambda (-\lambda x) = -\lambda^2 x$$
			\item 整理得 $x(1 + \lambda^2) = 0$。\begin{itemize}
				\item 若$x \neq 0$,则 $1 + \lambda^2 = 0 \implies \lambda = \pm \mathrm{i}$(复数解)。
				\item 若$x = 0$,则$y = -\lambda \cdot 0 = 0$,与$v$非零矛盾。
			\end{itemize}
			\item 验证$\lambda = \mathrm{i}$和$\lambda = -\mathrm{i}$对应的非零向量:
			\begin{itemize}
				\item 对$\lambda = \mathrm{i}$,由$y = -\mathrm{i} x$,特征向量为$(x, -\mathrm{i} x)$($x \neq 0$)。
				\item 对$\lambda = -\mathrm{i}$,由$y = \mathrm{i} x$,特征向量为$(x, \mathrm{i} x)$($x \neq 0$)。
			\end{itemize}
		\end{enumerate}
		所以在复数域$\mathbb{C}$中,$T$的特征值为$\mathrm{i}$和$-\mathrm{i}$。
	\end{enumerate}
\end{example}

根据上面的定义和例子,我们了解到特征值并不是唯一且一定拥有的,并我们要求$v\neq \boldsymbol{0}$这是因为每个标量$\lambda \in \mathbb{F}$都满足$T(\boldsymbol{0})=\lambda \boldsymbol{0}$。

根据一般的算子矩阵$\mathcal{T}$,下面推导矩阵的特征方程:设$\mathbf{A}$是一个$n\times n$方阵,$\mathbf{A}=\mathcal{M}(T)$,$\lambda$为$\mathbf{A}$的特征值,对应的非零特征向量$v$满足:$$\mathbf{A} v = \lambda v$$将等式改写为:$$\mathbf{A} v - \lambda v = \mathbf{0}$$引入单位矩阵$\mathbf{I}$以保持维度一致:$$(\mathbf{A} - \lambda \mathbf{I}) v = \mathbf{0}$$则非零解$v$存在的条件是系数矩阵的行列式为零:$$\det(\mathbf{A} - \lambda \mathbf{I}) = 0$$上述行列式方程即为矩阵$\mathbf{A}$的特征方程,展开后得到一个关于$\lambda$的多项式。

\begin{corollary}
	关于线性映射算子$T$的矩阵表示$\mathcal{M}(T)$的特征值$\lambda$满足$$\det(\mathbf{A} - \lambda \mathbf{I}) = 0$$
\end{corollary}

\begin{example}
	对于$2\times 2$矩阵 $ \mathbf{A} = \begin{pmatrix} a & b \\ c & d \end{pmatrix} $,求其特征方程,并求$a+b=3$,$ad-bc=2$时,矩阵$\mathbf{A}$的特征值.

	\tcblower
	\textcolor{purple}{\textbf{解}}: 对于$2\times 2$矩阵 $ \mathbf{A} = \begin{pmatrix} a & b \\ c & d \end{pmatrix} $特征方程为$$\det\left( \begin{pmatrix} a-\lambda & b \\ c & d-\lambda \end{pmatrix} \right) = (a-\lambda)(d-\lambda) - bc = 0$$展开后得到:$$\lambda^2 - (a+d)\lambda + (ad - bc) = 0$$带入$a+b=3$,$ad-bc=2$可得方程$$\lambda^2 - 3\lambda + 2 = 0$$因式分解可得$(\lambda-1)(\lambda-2)=0$解得$\lambda=1,2$故矩阵$\mathbf{A}$的特征值为$1$和$2$。
\end{example}

对于等式$T(v)=\lambda v$接下来我们逐步引出特征向量的概念,实际上这里的$v$就是特征向量。

\subsection{特征向量}

\begin{definition}{特征向量(Eigenvector)}
	设线性映射算子$T\in\mathcal{L}(V)$其中$V$为一个线性子空间,$\lambda \in \mathbb{F}$是算子$T$的特征值,若能满足等式$T(v)=\lambda v$且$v\neq 0,v\in V$则称$v$为$T$相对于$\lambda$的特征向量。
\end{definition}

下面我们介绍如何求解特征值所对应的特征向量,实际上它们与解线性方程组的方式一致,接下来请大家带着一个问题去看下面的例题,即:

\begin{ascolorbox1}{思考}
	特征向量是只有一个向量构成集合还是有很多个向量构成的集合?
\end{ascolorbox1}

\begin{example}
	算子$T\in\mathcal{L}(\mathbb{R}^2)$满足$T((x,y,z))=(y+z, x+z, x+y)$求;

	(1)算子$T$以标准基进行变换的矩阵表示$\mathcal{M}(T)$;

	(2)求算子$T$的所有特征值,并求出每个特征值对应的特征向量。
	\tcblower
	\textcolor{purple}{\textbf{解}}: 
	(1)算子$T$在标准基下的矩阵表示:
	
	标准基为$e_1 = (1, 0, 0)$,$e_2 = (0, 1, 0)$,$e_3 = (0, 0, 1)$。计算每个基向量在$T$作用下的结果:
	\begin{itemize}
		\item $T(e_1) = (0, 1, 1)$
		\item $T(e_2) = (1, 0, 1)$
		\item $T(e_3) = (1, 1, 0)$
	\end{itemize}
	因此,矩阵$\mathcal{M}(T)$为:$$\begin{pmatrix} 0 & 1 & 1 \\ 1 & 0 & 1 \\ 1 & 1 & 0 \end{pmatrix}$$

	(2)算子$T$的特征值及对应的特征向量:

	首先求解特征方程$\det(\mathcal{M}(T) - \lambda I) = 0$,其中矩阵$\mathcal{M}(T) - \lambda I$为:$$\begin{pmatrix} -\lambda & 1 & 1 \\ 1 & -\lambda & 1 \\ 1 & 1 & -\lambda \end{pmatrix}$$
	计算行列式:$$\det(\mathcal{M}(T) - \lambda I) = -\lambda^3 + 3\lambda + 2 = 0$$
	解得特征值为$\lambda = 2$和$\lambda = -1$(二重根)。
	\begin{itemize}
		\item 对于$\lambda = 2$,解方程组$(T - 2I)v = 0$,得到特征向量为所有非零标量倍的$(1, 1, 1)$即
		\item 对于$\lambda = -1$,解方程组$(T + I)v = 0$,得到特征向量为满足$x + y + z = 0$的所有非零向量,例如$(1, -1, 0)$和$(1, 0, -1)$。
	\end{itemize}
	计算行列式:$$\det(\mathcal{M}(T) - \lambda I) = -\lambda^3 + 3\lambda + 2 = 0$$解得特征值为$\lambda = 2$和$\lambda = -1$(二重根)。

	\begin{enumerate}
		\item 当特征值 $\lambda = 2$ 时$$\mathcal{M}(T) - 2I = \begin{pmatrix} -2 & 1 & 1 \\ 1 & -2 & 1 \\ 1 & 1 & -2 \end{pmatrix}$$对其进行使用初等变换$$ \begin{pmatrix} -2 & 1 & 1 \\ 1 & -2 & 1 \\ 1 & 1 & -2 \end{pmatrix}\xrightarrow{r_1\leftrightarrow r_2}\begin{pmatrix} 1 & -2 & 1 \\ -2 & 1 & 1 \\ 1 & 1 & -2 \end{pmatrix}\xrightarrow{r_2+2\times r_1}\begin{pmatrix} 1 & -2 & 1 \\ 0 & -3 & 3 \\ 3 & 1 & -2 \end{pmatrix}\xrightarrow{r_3-r_1}\begin{pmatrix} 1 & -2 & 1 \\ 0 & -3 & 3 \\ 0 & 3 & -3 \end{pmatrix}$$$$\xrightarrow{r_3+ r_2}\begin{pmatrix} 1 & -2 & 1 \\ 0 & -3 & 3 \\ 0 & 0 & 0 \end{pmatrix}\xrightarrow{r_2/3}\begin{pmatrix} 1 & -2 & 1 \\ 0 & -1 & 1 \\ 0 & 0 & 0 \end{pmatrix}$$还原为齐次线性方程组$$\left\{\begin{matrix} 
  		x_1-2x_2+x_3=0 \\  
  		-x_2+x_3=0 
		\end{matrix}\right. $$确定基础解系,设$x_1=1$则$x_2=1,x_3=1$由于$r(\mathcal{M}(T))=2$所以$n-r=1$故基础解系中一个$\mathcal{B}$共一个向量即$\mathcal{B}=\left\{ (1,1,1) \right\}$所以当特征值$\lambda=2$时,特征向量为$v=k(1,1,1),k\in \mathbb{R}$
		\item 当特征值 $\lambda = -1$时$$\mathcal{M}(T) + I = \begin{pmatrix} 1 & 1 & 1 \\ 1 & 1 & 1 \\ 1 & 1 & 1 \end{pmatrix}$$所有行相同,化为:$$\begin{pmatrix} 1 & 1 & 1 \\ 0 & 0 & 0 \\ 0 & 0 & 0 \end{pmatrix}$$其对应的方程为$$x+y+z=0$$最后令$x=0,y=1$则$z=-1$令$x=1,y=0$则$z=-1$那么基础解系$\mathcal{B}=\left\{ (0,1,-1),(1,0,-1) \right\}$特征向量为$v=s(-1, 1, 0) + t(-1, 0, 1)$
	\end{enumerate}
\end{example}

接下来回答上述的思考题:特征向量是只有一个向量构成集合还是有很多个向量构成的集合?答案是很多个向量构成的集合,因为最后此类问题求解特征向量都会转化为求齐次线性方程组,然而齐次线性方程组要么全为0解,要么有无数个非零解,全为0解的时,特征值为0或不存在故不存在特征向量。

